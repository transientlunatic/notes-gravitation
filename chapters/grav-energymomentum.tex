
\section{Dust, fluid, and flux}
\label{sec:dust-fluid-flux}

A fluid is a material which flows, that is, has forces perpendicular
to some imaginary surface which are much greater than the forces
parallel to it. A perfect fluid is the limit where the substance has
pressure but zero stress.

Dust is an idealised form of matter consisting of a collection of
non-interacting particles which are not moving relative to
one-another, and so the collection has zero pressure. Thus there is a
momentarily comoving reference frame (MCRF) with respect to which all
of the particles have zero velocity. 
If all of the particles have the same rest mass, $m$, but the cloud of
dust may have a varying mass density, $n$.

Transforming to a frame movig at a velocity $\vec{v}$ to the MCRF then
the stationary volume element $\Delta x \Delta y \Delta z$ will be
Lorentz contracted to $\Delta x' \Delta y' \Delta z' = (\Delta x /
\gamma) \Delta y \Delta z$, for relative motion along the $x$-axis,
increasing the number density to $n \gamma$, producing a flux through
the area $\Delta y \Delta z$. All the particles pass through $\Delta
y' \Delta z'$ in a time $\Delta t'$ for $\Delta x' = v \Delta t'$, so
the total number of particles is
\[ (\gamma n) (v \Delta t') \Delta y' \Delta z' \]
This produces an $x$-directed flux,
\begin{equation}
  \label{eq:76}
  N^x = \gamma n v^x
\end{equation}
or, defining a flux vector, $\vec{N}$, and letting $\vec{U} = (\gamma,
\gamma v^x, \gamma v^y, \gamma v^z)$ be the velocity 4-vector,
\begin{equation}
  \label{eq:77}
  \vec{N} = n \vec{U}
\end{equation}
In the MCRF $\vec{U} = (1, \vec{0})$, so $g(\vec{U}, \vec{U}) = -1$ so
\[ g(\vec{N}, \vec{N}) = N_{\alpha} N^{\alpha} = -n^2 \]
The components of the flux vector $\vec{N}$ in the frame are then
\[ \vec{N} = ( \gamma n, \gamma n \vec{v} ) \] Any function
$\phi(t,x,y,z)$ over spacetime defines a constant surface, and its
gradient $\of{\dd}{\phi}$ defines a normal to the surface. The unit normal gradient is defined
\[ \of{n} \equiv \frac{\of{\dd{}}{\phi}}{\abs{\of{\dd{}}{\phi}}} \]
Contracting this with the flux vector gives the flux across the
corresponding surface.

\section{The energy-momentum tensor}
\label{sec:energy-moment-tens}

Energy and mass are interconvertible (special relativity); for a dust
particle of mass $m$ the energy density of the dust (in the MCRF) is
$E = mn$. In a moving frame the number density becomes $n \gamma$, and
the energy of each particle is $\gamma m$, giving a total energy of
$\gamma^2 mn$ in a moving frame. The $\gamma^2$ term cannot be
generated by a simple Lorentz boost, so we require something of higher
order. Forming the $(2,0)$-tensor
\begin{equation}
  \label{eq:81}
  \ten{T} = \vec{p} \otimes \vec{N} = \rho \vec{U} \otimes \vec{U}
\end{equation}
with $\rho = mn$ the mass density of the dust. This is the
energy-momentum (stress-energy) tensor.  The components of the tensor
can be retrieved by contracting it with basis one-forms,
$\of{\omega}^{\alpha} = \of{\dd}x^{\alpha}$, 
\begin{equation}
  \label{eq:82}
  \tensor{T}{^{\alpha \beta}} = \ten{T}(\of{\dd}x^{\alpha}, \of{\dd}x^{\beta}) = \vec{p}(\of{\dd}x^{\alpha}) \cp \vec{N}(\of{\dd}x^{\beta})
\end{equation}
The $\mathbf{00}$ component, $T^{00}$ is the energy (flow of zeroth component of momentum across surface of constant time).\\
The $\mathbf{0i}$ component, $T^{0i} = \gamma m \times \gamma n v^i$ is the energy flux across a surface of constant $x^i$.\\
The $\mathbf{i0}$ component, $T^{i0} = p^i \times N^0 = m \gamma v^i
\times \gamma n$ is the flux of the $i$th component of momentum across
a surface of constant time into the future. This is the $i$th
component of momentum density. This is the same as the energy flux,
so \[ T^{i0} = T^{0i} \] The $\mathbf{ij}$ component is the flux of
$i$-momentum across a surface of constant $x^j$, and has the
dimensions of a pressure.

In general $\ten{T}$ is symmetric, $T^{\alpha \beta} = T^{\beta
  \alpha}$.

In a perfect fluid there is no preferred direction, so the spatial
part of $\ten{T}$ is proportional to the spatial part of the metric,
and there is no momentum transport perpendicular to the surface of a
fluid element. Thus
\begin{equation}
  \label{eq:83}
  T^{ij} = p \delta^{ij}
\end{equation}
for a perfect fluid.
Hence,
\begin{equation}
  \label{eq:84}
  \ten{T} = (\rho + p) \vec{U} \otimes \vec{U} + p \ten{g}
\end{equation}
Dust has no pressure, so in the MCRF it has
\begin{equation}
  \label{eq:85}
  \ten{T} = \diag(\rho, 0, 0, 0)
\end{equation}
The final property is the conservation law; if energy is conserved
then the energy-momentum entering an arbitrary volume must be equal to
that leaving it. Thus
\begin{equation}
  \label{eq:86}
  \pdv{x^0} T^{\alpha 0} + \pdv{x^1} T^{\alpha 1} + \pdv{x^2} T^{\alpha 2} + \pdv{x^3} T^{\alpha 3} = 0
\end{equation}
That is
\begin{equation}
  \label{eq:87}
  \ten{T}{^{\alpha \beta}_{, \beta}} = 0
\end{equation}
Similarly
\begin{equation}
  \label{eq:88}
  \tensor{N}{^{\alpha}, \alpha} = (n U^{\alpha})_{, \alpha} = 0
\end{equation}

% \section{Maxwell's Equations}
% \label{sec:maxwells-equations}

% A covariant form of the Maxwell equations is the Faraday tensor,
% \begin{equation}
%   \label{eq:89}
%   \ten{F} =
%   \begin{bmatrix}
%     0 & E^x & E^y & E^z \\
%  -E^x & 0   & B^z & - B^y \\
%  -E^y & -B^z & 0 & B^x \\
%  -E^z & B^y & -B^x & 0
%   \end{bmatrix}
% \end{equation}


%%% Local Variables: 
%%% mode: latex
%%% TeX-master: "../project"
%%% End: 
