
In Newtonian theory the stability of stars is described in terms of
hydrostatic equilibrium. In the framework of General Relativity this
is similar, leading to the Oppenheimer-Volkoff equation.

\section{Components of the Einstein tensor}
\label{sec:comp-einst-tens}

The stellar interior will have a metric of the form
\[ \dd{s}^2 = -e^\nu \dd{t}^2 + e^{\lambda} \dd{r}^2 + r^2 \qty(
\dd{\theta}^2 + \sin[2](\theta) \dd{\phi}^2 )\] where $\nu$ and
$\lambda$ are functions of $r$. The Ricci tensor will have components
\begin{subequations}
\begin{equation}
  \label{eq:193}
  R_{tt} = \half e^{\nu-\lambda} \qty( \nu'' + \half \nu'^2 - \half \nu' \lambda' + \frac{2}{r} \nu')
\end{equation}
\begin{equation}
  \label{eq:201}
  R_{rr} = - \half \qty( \nu'' + \half \nu'^2 - \half \nu' \lambda' - \frac{2}{r} \lambda')
\end{equation}
\begin{equation}
  \label{eq:202}
  R_{\theta \theta} = 1 - e^{- \lambda} \qty( 1 + \frac{r}{2} ( \nu' - \lambda') )
\end{equation}
\begin{equation}
  \label{eq:203}
  R_{\phi \phi} = R_{\theta \theta} \sin[2](\theta)
\end{equation}
\end{subequations}
with all other components zero. The metric is orthogonal, so
\begin{subequations}
  \begin{align}
    g^{tt} &= - e^{- \nu} \\ g^{rr} &= e^{- \lambda} \\ g^{\theta \theta} &= r^{-2} \\ g^{\phi \phi} &= (r^2 \sin[2](\theta) )^{-1}
  \end{align}
\end{subequations} 
again, with all other terms zero.
The Ricci and metric tensors are orthogonal, so the Ricci scalar has the expression
\begin{equation}
  \label{eq:204}
  R = g^{\mu \nu} R_{\mu \nu} = g^{tt}R_{tt} + g^{rr} R_{rr} + g^{ \theta \theta} R_{\theta \theta} + g^{\phi \phi} R_{ \phi \phi}
\end{equation}
after making the appropriate substitutions, we find
\begin{equation}
  \label{eq:205}
  \begin{split}
    R = - e^{-\lambda} \qty[ \qty( \nu'' + \half \nu'^2 - \half \nu' \lambda' ) + \frac{\nu' - \lambda'}{r}]\\
{}+\frac{2}{r^2} \qty[1 - e^{- \lambda} \qty( 1+ \frac{(\nu' - \lambda')r}{2})]
  \end{split}
\end{equation}
The fully covariant Einstein tensor is
\[ G_{\mu \nu} = R_{\mu \nu} - \half g_{\mu \nu} R \]
so
\begin{subequations}
  \begin{align}
    G_{tt} &= \frac{e^{\nu}}{r^2} \qty[ 1+ e^{-\lambda} (r \lambda' -1) ] \\
G_{rr} &= \frac{\nu'}{r} - \frac{e^{\lambda}}{r^2} \qty( 1 - e^{- \lambda}) \\
G_{\theta \theta} &= r^2 e^{- \lambda} \qty[ \frac{\nu''}{2} + \frac{\nu'^2}{4} - \frac{\nu' \lambda'}{4} + \frac{\nu'-\lambda'}{2r}] \\
G_{\phi \phi} &= \sin[2](\theta) G_{\theta \theta}
  \end{align}
\end{subequations}

\section{Components of the energy-momentum tensor}
\label{sec:comp-energy-moment}

For a perfect fluid the energy-momentum tensor, in fully covariant
form has components
\[ T^{\mu \nu} = (\rho + P) u^{\mu} u^{\nu} + P g^{\mu \nu} \] for
$\rho$ the mass-energy density, and $P$ the pressure, while $u^{\mu}$
are the four velocity components of a fluid element. Seeking a static
solution has $u^{i} = 0, i = \set{1,2,3}$, so from the geodesic
equation
\begin{equation}
  \label{eq:206}
  g_{tt}(u^t)^2 = -1 \implies u^t = e^{- \frac{\nu}{2}}
\end{equation}
thus
\begin{subequations}
  \begin{align}
    T_{tt} &= \rho e^{\nu} \\
T_{rr} &= P e^{\lambda} \\
T_{\theta \theta} &= P r^2 \\
T_{\phi \phi} &= P r^2 \sin[2](\theta)
  \end{align}
\end{subequations}

\section{Einstein's equations}
\label{sec:einsteins-equations-1}

In fully covariant form
\[ G_{\mu \nu} = 8 \pi T_{\mu \nu} \]
so
\begin{subequations}
  \begin{align}
\label{eq:207}
    \frac{e^{\nu}}{r^2} \qty[  1+ e^{-\lambda} (r \lambda' -1) ] &= 8 \pi \rho e^{\nu} \\
\label{eq:208}
    \frac{\nu'}{r} - \frac{e^{\lambda}}{r^2} (1-e^-\lambda) &= 8 \pi P e^{\lambda} \\
\label{eq:209}
r^2 e^{-\lambda} \qty[ \frac{\nu''}{2} + \frac{\nu'^2}{4} - \frac{\nu' \lambda'}{4} + \frac{\nu'-\lambda'}{2r}] &= 8 \pi P r^2
  \end{align}
\end{subequations}

\section{Solution of the first Einstein equation}
\label{sec:solut-first-einst}

By cancelling the $e^{\nu}$ term in equation (\ref{eq:207}), and
rearranging,
\begin{equation}
  \label{eq:210}
  \dv{r} \qty[ r(1-e^{-\lambda})] = 8 \pi \rho r^2
\end{equation}
Then introducing the mass function, $m(r)$, defined as
\begin{equation}
  \label{eq:211}
  \dv{m}{r} = 4 \pi \rho r^2 = \half \dv{r} \qty[r (1-e^{-\lambda})]
\end{equation}
and integrating,
\begin{equation}
  \label{eq:212}
  r(1-e^{-\lambda}) = 2m +C 
\end{equation}
where $C$ is a constant of integration, equal to zero unless the star
is singular at $r=0$. Thus
\begin{equation}
  \label{eq:213}
  e^{-\lambda} = 1 - \frac{2m}{r}
\end{equation}

\section{The Oppenheimer-Volkoff equation}
\label{sec:oppenh-volk-equat}

Rearranging equation (\ref{eq:208}), with some straight-forward
algebra,
\begin{equation}
  \label{eq:214}
  \dv{\nu}{r} = e^{\lambda} \qty[8 \pi P r + \frac{1}{r} (1-e^{-\lambda})] = 2 \qty[\frac{4 \pi P r^3 + m}{r(r-2m)}]
\end{equation}
Then, considering conservation of mass-energy, $\tensor{T}{^{\alpha
    \beta}_{;\beta}} = \qty[(\rho +P) u^{\alpha}u^{\beta} + P
g^{\alpha \beta}]_{;\beta}= 0$, then
\begin{equation}
  \label{eq:215}
\begin{split}
  (\rho + P)_{;\beta} u^{\alpha} u^{\beta} + (\rho+P)(u^{\alpha})_{; \beta} u^{\beta} \\ + (\rho+P) u^{\alpha}(u^{\beta})_{; \beta} + P_{,\beta} g^{\alpha \beta} + Pg^{\alpha \beta}_{; \beta} = 0
\end{split}
\end{equation}
This is infact a set of four equations, as $\alpha$ is a free index,
so with just $\alpha \equiv r$,
\begin{equation}
  \label{eq:215}
\begin{split}
  (\rho + P)_{;\beta} u^{r} u^{\beta} + (\rho+P)(u^{r})_{; \beta} u^{\beta} \\ + (\rho+P) u^{r}(u^{\beta})_{; \beta} + P_{,\beta} g^{r \beta} + Pg^{r \beta}_{; \beta} = 0
\end{split}
\end{equation}
But since $u^r=0$ two of these terms vanish, and the derivatives of
the metric tensor are all zero, so the last term drops out too. We
then have the simplification
\begin{equation}
  \label{eq:216}
  ( \rho +P)(u^r)_{;t} u^t + \dv{P}{r} g^{rr} = 0
\end{equation}
then
\begin{equation}
  \label{eq:217}
  u^r_{;t} = \Gamma^r_{tt} u^t = \half \nu' e^{\nu-\lambda} e^{-\nu/2} = \half e^{-\lambda} \nu' e^{\nu/2}
\end{equation}
Thus
\begin{equation}
  \label{eq:218}
  \half(\rho + P) e^{-\lambda} \dv{\nu}{r} + e^{- \lambda} \dv{P}{r} = 0 \implies \dv{\nu}{r} = - \frac{2}{(\rho+P)} \dv{P}{r}
\end{equation}
and using this to eliminate $\nu'$ from equation (\ref{eq:214}),
\begin{equation}
  \label{eq:219}
  \dv{P}{r} = - \frac{(\rho + P)(4 \pi P r^3 + m)}{r(r-2m)}
\end{equation}
which is the \emph{Oppenheimer-Volkoff} equation. In the weak-field limit, $P \ll \rho$ implies $4 \pi P r^3 \ll m$, and the metric will be almost flat, so $m \ll r$, so this simplifies to
\begin{equation}
  \label{eq:220}
  \dv{P}{r} = - \frac{\rho m}{r^2}
\end{equation}
which is the Newtonian hydrostatic equilibrium equation.

\section{Solving the O-V equation}
\label{sec:solv-oppenh-volk}

There are three unknown functions in the O-V equation, $P(r)$,
$\rho(r)$, and $m(r)$, but the latter two are related, so we need an
additional relation, an equation of state, to link all three, in the form
\[ P(r) = P(\rho(r)) \]
For a fluid which is in local thermodynamic equilibrium there is always a relation between pressure, density, and entropy, of the form
\[ P=P(\rho,S) \] In most astrophysical situations we can regard $S$
as constant. In practice, to solve this system we need boundary
conditions.

Take $P=P_0$ and $m=0$ at $r=0$, and integrate outwards to $P=0$ at
the surface of the star, where $r=R$ and $m=M$, where $M$ is the mass
constant in the exterior Schwarzschild metric. This then allows us to
find $\nu$ and $\lambda$ to form a complete expression for the metric
inside the star. The effect of GR compared to Newtonian mechanics will
be to steepen the pressure gradient within the star.

\section[Constant density star solution]{An exact solution for a star with constant density}
\label{sec:an-exact-solution}

Suppose that the density is constant, and $\rho=\rho_0$ (implying,
rather concerningly, an infinite sound speed!), and integrating
equation (\ref{eq:211}), and retrieve
\begin{equation}
  \label{eq:221}
  m(r) = \frac{4}{3} \pi \rho_0 r^3
\end{equation}
This can be substituted into the O-V equation, giving
\begin{equation}
  \label{eq:222}
  \dv{P}{r} = - \frac{4}{3} \pi r \frac{(\rho_0+P)(\rho_0 + 3 P)}{(1 - \frac{8 \pi \rho_0 r^2}{3})}
\end{equation}
thus
\begin{align*}
  \label{eq:223}
  \frac{\dd{P}}{(\rho_0 +P)(\rho_0 + 3P)} &= \frac{1}{2 \rho_0} \qty[\frac{3 \dd{P}}{(\rho_0 + 3P)} - \frac{\dd{P}}{(\rho_0 +P)}] \\ &= - \frac{4 \pi}{3} \frac{r \dd{r}}{\qty(1 - \frac{8 \pi \rho_0 r^2}{3})}
\end{align*}
and integrating both sides,
\begin{equation}
  \label{eq:224}
  \log( \rho_0 + 3P) - \log(\rho_0 + P) = \half \log(1 - \frac{8 \pi \rho_0 r^2}{3}) + C
\end{equation}
for a constant $C$, which can be written
\begin{equation}
  \label{eq:225}
  \frac{\rho_0 + 3P}{\rho_0+P} = A \qty( 1 - \frac{8 \pi \rho_0 r^2}{3})^{1/2}
\end{equation}
when $r=0$ we have $P=P_0$, so we can express $A$ in terms of density and central pressure,
\[ A = \frac{\rho_0 + 3 P_0}{\rho_0 +P_0} \]
Then
\begin{align}
  \label{eq:226}
  \frac{\rho_0 + 3P}{\rho_0 + P} &= \frac{\rho_0 + 3P_0}{\rho_0 + P_0}  \qty( 1 - \frac{8 \pi \rho_0 r^2}{3})^{1/2} \nonumber\\
&=  \frac{\rho_0 + 3P_0}{\rho_0 + P_0} \qty( 1 - \frac{2m}{r})^{1/2}
\end{align} at the surface of the star $P=0$ so the left hand reduces
to $1$, so
\begin{equation}
  \label{eq:227}
  \frac{\rho_0 + 3P_0}{\rho_0 + P_0} \qty( 1 - \frac{2m}{r})^{1/2} = 1
\end{equation}
for $M$ the Schwarzschild mass, and $R$ is the coordinate radius of
the star. We can obtain an expression for $P$ as a function of $r$ by rearranging,
\begin{equation}
  \label{eq:228}
  P_0 = \frac{\rho_0 \qty[ 1 - \qty( 1 - \frac{2M}{R} )^{1/2}]}{3 \qty( 1 - \frac{2M}{R})^{1/2} - 1}
\end{equation}

\section{Buchdahl's Theorem}
\label{sec:buchdahls-theorem}

From equation (\ref{eq:228}), it can be seen that as $P_0 \to \infty$
when $3 \qty(1- \frac{2M}{R})^{1/2} \to 1$, that is, when $M/R \to
4/9$. Clearly there can be no static star with uniform density with a
radius smaller than $9M/4$, as they would require an infinite internal
pressure. If we require the exterior metric to be well-behaved then
stars with a radius less than $2M$ should be excluded, as the
Schwarzschild metric misbehaves at this point, with timelike intervals
becoming spacelike, and vice versa.

As a result we can exclude the possibility of a uniform static star
with these properties, and \emph{Buchdahl's theorem} is a rigorous
proof of this.



%%% Local Variables: 
%%% mode: latex
%%% TeX-master: "../project"
%%% End: 
