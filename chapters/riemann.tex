
\section{Parallelism on curved surfaces}
\label{sec:parall-curv-surf}

A differentiable manifold has no concept of parallelism between
vectors defined at two points. An affine connection provides a rule to
provide such a concept. Consider transporting a vector across the
surface of a sphere from a pole to the equator, keeping it tangent to
the sphere at all times, without rotating it as it is
transported. Then transporting the vector 90 degrees around the
equator, and returning to the pole, we find the vector rotated by 90
degrees; clearly there is no global concept of parallelism. So we
define a rule for conducting this transport in a parallel fashion.

\section{Covariant Derivative}
\label{sec:covariant-derivative}

Suppose we have a curve, $C$, and a connection. The tangent to $C$ is
$U = \dv{\lambda}$, and we have a vector $V$ from the tangent space
$T_P$ at $P$. By parallel-transporting $V$ along the curve we generate
a new vector field, and as this does not change along the curve, the
curve's derivative with respect to the field is zero, so
\begin{equation}
  \label{eq:41}
  \nabla_U V = 0
\end{equation}
If $W$ is a vector field defined everywhere along $C$ then its
covariant derivative can be defined in a similar way to Lie
derivatives, 
\begin{equation}
  \label{eq:42}
  ( \nabla_U W)_P = \lim_{\epsilon \to 0} \frac{W_{\lambda_0 + \epsilon}(\lambda_0) - W(\lambda_0)}{\epsilon}
\end{equation}
where $W_{\lambda_0+\epsilon}$ is the parallel transported field which
equals $W$ at the point $\lambda_0+\epsilon$. The operation of
transportation is not the same as the dragging-back of the Lie
derivative, which requires the whole congruence, and not just the
curve. The $\nabla_U$ object is a differential operator:
\begin{subequations}
\begin{equation}
  \label{eq:43}
  \nabla_U (fW) = f \nabla_U W + W \nabla_U f = f \nabla_U W + W \dv{f}{\lambda}
\end{equation}
It can be extended to tensors of arbitrary type by the Leibniz rules,
\begin{equation}
  \label{eq:44}
  \nabla_U (A \otimes B) = (\nabla_U A) \otimes B + A \otimes (\nabla_U B)
\end{equation}
\begin{equation}
  \label{eq:45}
  \nabla_U \langle \of{\omega}, A \rangle = \langle \nabla_U \of{\omega}, A \rangle + \langle \of{\omega}, \nabla_U A \rangle
\end{equation}
\end{subequations}

Now suppose the parameter along the curve is changed from $\lambda$ to
$\mu$, so the new tangent is $gU$, for $g = \dv{\lambda}{\mu}$, then
\begin{subequations}
\begin{equation}
  \label{eq:46}
  \nabla_{gU} W = g \nabla_U W
\end{equation}
We also require
\begin{equation}
  \label{eq:47}
(\nabla_U W)_P +(\nabla_V W)_P = (\nabla_{U+V} W)_P
\end{equation}
\end{subequations}
The fact that $\nabla W$ is a tensor field---the gradient of $W$
allows the curve can be removed from the definition of the covariant
derivative.

\section{Covariant derivatives of a basis}
\label{sec:covar-deriv-basis}

Any tensor may be described as a linear combination of basis tensors,
which can be derived from basis vectors, and the connection can be
retrieved from knowledge of the gradients of the basis vectors, so we
define
\begin{equation}
  \label{eq:48}
  \nabla_{e_i} e_j = \tensor{\Gamma}{^k_{ji}} e_k
\end{equation}
with the symbol $\tensor{\Gamma}{^k_{ji}}$ the Christoffel symbols;
for fixed $(i,j)$ this is the $k$th component of the vector field
$\nabla_i e_j$, defining $\nabla_i \equiv \nabla_{e_i}$. These
completely determine the connection, although are not the components
of a tensor.

Knowing the derivatives of the basis vectors, for an arbitrary tensor, with $U=\dv{\lambda}$, then
\begin{equation}
  \label{eq:49}
  \nabla_U V = U^i \nabla_{e_i}(V^j e_j) = U^i( \nabla_{e_i} V^j )e_j + U^i V^j \nabla_{e_i} e_j
\end{equation}
given that $V^j$ is a function, $U^i \nabla_i (V^j) =
\dv{V^j}{\lambda}$, so
\begin{equation}
  \label{eq:50}
  \nabla_U V = \dv{V^j}{\lambda} e_j + U^i V^j \tensor{\Gamma}{^k_{ji}}e_k = \qty( \dv{V^j}{\lambda} + \tensor{\Gamma}{^j_{ki}} V^k U^i ) e_j
\end{equation}
If $e_i = \dv{\mu}$ then $\nabla_i (V^j) = \dv{V^j}{\mu}$, with $V^j$
function along the curve with parameter $\mu$, if $e_i = \pdv{x^i}$ is
a coordinate vector, then
\begin{equation}
  \label{eq:51}
  \nabla_i V^j = \pdv{x^i} V^j = \tensor{V}{^j_{,i}}
\end{equation}
which introduces the comma notation for forms. Likewise we can
introduce the semicolon notation for a covariant derivative,
\begin{equation}
  \label{eq:52}
  \tensor{(\nabla V)}{^j_i} = \tensor{V}{^j_{,i}} + \tensor{\Gamma}{^j_{ki}} V^k = \tensor{V}{^j_{;i}}
\end{equation}

\section{Torsion}
\label{sec:torsion}

A connection is symmetric if the quantities $\comm{U}{V}$ and
$\nabla_UV - \nabla_V U$ are equal.
\[ \comm{U}{V} = \nabla_U V- \nabla_V U \]

For a non-symmetric connection we can define the torsion,
\begin{equation}
  \label{eq:53}
  \tensor{T}{^k_{ji}} e_k = \nabla_{e_j} e_j - \nabla_{e_j} e_i - \comm{e_i}{e_j}
\end{equation}

The symmetric part of the connection can be defined for any
connection:
\begin{equation}
  \label{eq:54}
  \tensor{\Gamma}{^k_{\rm (S)}_{ij}} = \tensor{\Gamma}{^k_{ij}} - \half \tensor{T}{^k_{ij}}
\end{equation}

\section{Geodesics}
\label{sec:geodesics}

A geodesic is a curve which parallel transports its own tangent
vector. The geodesic equation is 
\begin{equation}
  \label{eq:55}
  \nabla_U U = 0
\end{equation}
For $\lambda$ the parameter of the curve, and $\set{x^i}$ a coordinate system this becomes
\begin{equation}
  \label{eq:56}
  \dv{U^i}{\lambda} + \tensor{\Gamma}{^i_{jk}} U^j U^k = \dv[2]{x^i}{\lambda} + \tensor{\Gamma}{^i_{jk}} \dv{x^j}{\lambda} \dv{x^k}{\lambda} = 0
\end{equation}
If the parameter of a curve is changed from $\lambda$ to $\mu = a
\lambda +b$, for $a$, $b$ constants, we find the parameter $\mu$ is
also a solution of the geodesic equation, and so a geodesic's
paramater is \emph{affine}.

Only the symmetric part of the connection contributes to the geodesic equation, and this allows the effect of torsion to be seen geometrically; a vector parallel transported along a geodesic with torsion will be rotated relative to nearby geodesics.

\section{Normal coordinates}
\label{sec:normal-coordinates}

The geodesics through a point $P$ provide a bijection of the
neighbourhood of $P$ onto the origin of its tangent space, $T_P$,
because each element in $T_P$ defines a unique geodesic through $P$,
so each vector in $T_P$ can be associated with a point which is an
affine parameter, $\Delta \lambda =1$ along the curve from $P$.

With this map and an arbitrary basis for $T_P$ we have the normal
coordinates of a point $Q$. Such a map will normally only be bijective
in a neighbourhood of $P$.

A useful property of normal coordinates is that
$\tensor{\Gamma}{^i_{jk}} = 0$ at $P$, but not necessarily anywhere
else, so the derivative is non-vanishing.

\section{The Riemann tensor}
\label{sec:riemann-tensor}

The Riemann tensor is a multiplicative operator which is also a
$(1,1)$-tensor, defined

\newcommand{\rie}{\mathbf{\mathsf{R}}}
\begin{definition}[The Riemann tensor]
  \begin{equation}
    \label{eq:59}
    \comm{\nabla_U}{\nabla_V} - \nabla_{\comm{U}{V}} \equiv \rie(U,V)
  \end{equation}
Its components are defined
\[ \tensor{\rie}{^l_{kij}} e_l = \comm{\nabla_i}{\nabla_j} e_k - \nabla_{\comm{e_i}{e_j}} e_k \]
\end{definition}
 
We can see a geometrical interpretation of $\rie$ by considering a
vector field, $A$, defined along a curve with tangent $U$. Parallel
transporting $A$ to $Q$ produces a vector in $T_P$, $A(Q \to P)$ is
the \emph{image} of $P$ at $A(Q)$; if $A$ and $U$ are analytic
\[ A(Q \to P) = \exp( \lambda \nabla_U ) \eval{A}_P \] With two
commuting congruences, $U = \dv*{\lambda}$ and $V = \dv*{\mu}$, and a
vetor is parallel transported from $R$ to $Q$ along $V$ we get a
vector at $Q$,
\[ A(R \to Q) = \exp(\mu \nabla_V) \eval{A}_Q \] For $\mu$ the
parameter distance from $Q$ to $R$. Transporting this vector to $P$ we
get
\[ A(R \to Q \to P) = \exp(\lambda \nabla_U) \exp(\mu \nabla_V)
\eval{A}_P \] We can move the vetor around the other two edges of a
square to get to the same point, and find that the two methods produce
different results, the difference being
\begin{equation}
  \label{eq:61}
  \delta A = \lambda \mu \comm{\nabla_U}{\nabla_V} A + \mathcal{O}(3)
\end{equation}
which is the Riemann tensor. This is the change in $A$ produced if it
is parallel transported about a closed loop, which is the area,
$\lambda \mu$ of the loop times the Riemann tensor.
\[ \delta A^i = \lambda \mu \tensor{\rie}{^i}{_{jkl}}A^j U^k V^l \]

Geodesics which start parallel do not stay parallel---this is geodesic
deviation. Consider a congruence of geodesics, with tangent $U$, and a
connecting vector, $\xi$ which is Lie dragged by the congruence. The
measure of the change in the geodesics come from the seconds
derivative of the vector $\xi$, $\nabla_U \nabla_U \xi$ which measures
how the separation of geodesics changes. Then
\begin{align*}
  \nabla_U \nabla_U \xi &= \nabla_U \qty( \ld{U}{\xi} + \nabla_{\xi}U) \\
  &= \nabla_U \nabla_{\xi} U \\ &= \comm{\nabla_U}{\nabla_{\xi}} U +
  \nabla_{\xi} \nabla_U U
\end{align*}
The last term disappears since $U$ is a geodesic, and so
\begin{subequations}
\begin{equation}
  \label{eq:62}
  \nabla_U \nabla_U \xi = \rie(U, \xi) U
\end{equation}
which has a component form
\[ \qty( \tensor{\xi}{^i_{;j}} U^j )_{;k} U^k = \tensor{R}{^i_{jkl}}
U^j U^k \xi^l \] The left-hand side can be simplified, as
$\tensor{U}{^j_{;k}} U^k = 0$, and so
\begin{equation}
  \label{eq:63}
  \tensor{\xi}{^i_{;j;k}}U^j U^k = \tensor{R}{^i_{jkl}}U^j U^k \xi^l
\end{equation}
\end{subequations}

We normally require a manifold with a metric and a connection to have
some compatibility demands. For example, the divergence of a vector field; the covariant divergence is defined
\[ \div V \equiv \nabla_i V^i \]
and the volume-form divergence is 
\[ \ld{V}{\of{\omega}} = \qty(\operatorname{div}_{\of{\omega}} V)
\of{\omega} \] Then $\nabla$ and $\of{\omega}$ are compatible if
$\operatorname{div}$.  \newcommand{\met}{\mathbf{\mathsf{g}}}

If the manifold has a metric tensor, $\met$, then an inner product is
defined at any point, and the connection and metric are compatible if
parallel transport preserves the inner product of two vectors. Thus a
metric determines its compatible symmetric connection, a \emph{metric
  connection}.

We can express the compatibility of the metric and the connection by
asserting that they are compatible iff
\begin{equation}
  \label{eq:64}
  \nabla \met = 0
\end{equation}
and
\begin{equation}
  \label{eq:65}
  \tensor{\Gamma}{^i_{jk}} = \half g^{il} (g_{lj,k} + g_{lk,j} - g_{jk,l})
\end{equation}

\section{Metric Connections}
\label{sec:metric-connections}

Metric connections have additional properties which general
connections do not have, these can be derived in a normal coordinate
system. The earlier conditions in equations \eqref{eq:64} and
\eqref{eq:65} there is the implication
\begin{equation}
  \label{eq:66}
  \tensor{\Gamma}{^i_{jk}} = 0 \text{ at } P \iff g_{lm,n} = 0 \text{ at } P
\end{equation}
We now have the implication, that in normal coordinates at a point
$P$,
\begin{equation}
  \label{eq:67}
  \tensor{R}{_{ijkl}} \equiv \tensor{g}{_{im}} \tensor{R}{^m_{jkl}} 
  = \half \qty( \tensor{g}{_{il,jk}} - \tensor{g}{_{ik, jl}}+ \tensor{g}{_{jk, il}} - \tensor{g}{_{jl, ik}})
\end{equation}
which has the property
\begin{equation}
  \label{eq:68}
  \tensor{R}{_{ijkl}} = \tensor{R}{_{klij}}
\end{equation}

We can define two new quantities, the Ricci tensor,
\begin{equation}
  \label{eq:69}
  \tensor{R}{_{kl}} = \tensor{R}{^i_{kil}}
\end{equation}
which is symmetric, and the Ricci scalar,
\begin{equation}
  \label{eq:70}
  R = \tensor{g}{^{kl}} \tensor{R}{_{kl}}
\end{equation}
And, given the Bianchi identities,

\[ \tensor{R}{^i_{j[il;m]}} = 0 \qquad \tensor{g}{^{jl}}\tensor{R}{^i_{j[il;m]}} = 0 \]
then
\begin{equation}
  \label{eq:72}
  \qty( \tensor{R}{^{ij}} - \half R g^{ij} )_{;j} = 0
\end{equation}

The Weyl tensor can then be defined
\begin{equation}
  \label{eq:73}
  \tensor{C}{^{ij}_{kl}} = \tensor{R}{^{ij}_{kl}} - 2 \tensor{\delta}{^{[i}_{[k}} \tensor{R}{^{j]}_{l]}} + \frac{1}{3} \tensor{\delta}{^{[i}_{[k}} \tensor{\delta}{^{j]}_{l]}} R
\end{equation}

%%% Local Variables: 
%%% mode: latex
%%% TeX-master: "../project"
%%% End: 
