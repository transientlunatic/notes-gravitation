A congruence is a set of curves which fill a manifold, or some part of
it without intersecting. Each point is on only one curve, and as each
curve is a 1D collection of points we can assemble the curves into an
($n-1$)-dimensional manifold. The congruence thus provides a mapping
from the manifold back onto itself. If the parameter on the curves is
$\lambda$ then any small number $\Delta \lambda$ defines a mapping in
which each point is moved $\Delta \lambda$ further along the same
curve. If this map exists for all $\Delta \lambda$ then there is a 1D
differentiable family of such mappings forms a Lie Group with the
composition law $\Delta \lambda_1 + \Delta \lambda_2$. Such a mapping
is a Lie dragging.

For a function $f$ defined on a manifold the Lie dragging defines a
function $f^{*}_{\Delta\lambda}$ such that if a point $P$ was mapped
to a point $Q$, after moving along the curve the new
field $f^{*}_{\Delta \lambda}$ has the same value at $Q$ as $f$ had at
$P$, that is
\[ f(P) = f^{*}_{\Delta \lambda} (Q) \] If $f^{*}_{\Delta \lambda} (Q)
= F(Q)$ for all $Q$ the function is invariant under the mapping. If it
is invariant under all $\Delta \lambda$ then it is Lie Dragged by the
field $\df{\lambda}$.

In the limit of infinitessimal $\Delta \lambda$ and infinitessimal
separation between two curves, $(1)$ and $(2)$, if $\dv{\mu}$ at the
point $P$ stretches from $P$ to $R$ along $(A)$, then
$\dv{\mu^{*}_{\Delta \lambda}}$ will stretch from $Q$ to $S$ on
$(A')$. If $\dv{\mu}$ is Lie dragged, then it will also stretch from
$Q$ to $S$, implying $\comm{\dv{\lambda}}{\dv{\mu}} = 0$. Thus a
vector field is Lie dragged if its Lie Bracket with the dragging field
vanishes.

\section{Lie Derivatives}
\label{sec:lie-derivatives}

The concept of dragging permits the definition of a derivative along
the congruence. There is a problem of defining distance in vector and
tensor fields; for a curve between points the distance is the
difference in the curves parameter for the two points. Another
onsideration is whether two vectors at different points are parallel
or not. On a simple manifold there is no sense of this question, as
there is no way to move vecors around in a parallel manner, and for
this we need an affine connection. However, the notion of congruence
can provide a substitute; to compare vectors at points $\lambda$ and
$\lambda + \Delta \lambda$ on a given curve we can Lie drag the vector
at $\lambda + \Delta \lambda$ back to $\lambda$, defining a new vector
at $\lambda$, which can be subtracted from the old one, giving the
difference between them which is unique given the congruence. This allows the definition of a derivative of the form
\begin{equation}
  \label{eq:18}
  \begin{split}
     \lim_{\Delta \lambda \to 0} \frac{f^{*}(\lambda_0) - f(\lambda_0)}{\Delta \lambda} = \lim_{\Delta \lambda \to 0} \frac{f(\lambda_0 + \Delta \lambda) - f(\lambda_0)}{\Delta \lambda} \\= \qty[ \dv{f}{\lambda}]_{\lambda_0}
   \end{split}
\end{equation}
We can define an operator for this derivative, $\ld{V}{}$ with $V$ the
vector field generating the mappings, so
\begin{equation}
  \label{eq:19}
  \ld{V}{f} = V(f) = \dv{f}{\lambda}
\end{equation}
Carrying out the same procedure on a field $U = \dv{\mu}$, and
considering the definition of a vector in terms of its effect on
functions, so we define an arbitrary function $f$. At $\lambda_0$ the
field $U$ gives the derivative $\qty(\dv{f}{\mu})_{\lambda_0}$ and at
$\qty(\dv{f}{\mu})_{\lambda_0 + \Delta \lambda}$; by dragging
$U(\lambda_0 + \Delta \lambda)$ we get a new field
$U^{*}=\dv{\mu^{*}}$, then $\comm{U^{*}}{V} = 0$, and $U^{*}(\lambda_0
+ \Delta \lambda) = U(\lambda_0+\Delta \lambda)$. The vanishing
commutator implies
\begin{equation}
  \label{eq:20}
  \dv{\lambda} \dv{\mu^{*}} f = \dv{\mu^{*}} \dv{\lambda} f
\end{equation}
We then define the Lie derivative $\ld{V}{U}$ as the vector field
operating on $f$ to give
\begin{equation}
  \label{eq:21}
  \qty[ \ld{V}{U}](f) = \lim_{\Delta \lambda \to 0} \qty( \dv{\lambda} \dv{\mu} f - \dv{\mu^{*}} \dv{\lambda} f)
\end{equation}
This is true for all $f$, so
\begin{equation}
  \label{eq:22}
  \ld{V}{U} = \dv{\lambda} U - \dv{\mu} V = \comm{V}{U}
\end{equation}

\subsection{Lie Derivatives of a One-form}
\label{sec:lie-derivatives-one}

Fields of one-forms and tensors are defined in terms of scalar
functions and vector fields, so a definition of a Lie derivative can
also be found for a one-form. A one-form field is said to be Lie
dragged if its value on any dragged vector field is constant, so the
Lie derivative of $\covec{\omega}$ along a vector field $V$ is defined
\begin{equation}
  \label{eq:23}
  \ld{V}{\of{\omega}} = \qty(\ld{V}{\of{\omega}})(W) + \of{\omega} \qty(\ld{V}{W})
\end{equation}
extending this to tensors of higher order,
\begin{equation}
  \label{eq:24}
  \ld{V}{(\ten{A} \otimes \ten{B})} = (\ld{V}{\ten{A}}) \otimes \ten{B} + \ten{A} \otimes (\ld{V}{\ten{B}})
\end{equation}

\section{Submanifolds}
\label{sec:submanifolds}

\textbf{A submanifold}, is a smooth subset of another manifold $M$. In
Euclidean 3-space surfaces and curves are submanifolds. Thus, An
$m$-dimensional submanifold, $S$, of an $n$-dimensional manifold $M$
is a set of points with the property that in some open neighbourhood
in $M$ of any point $P$ of $S$ there exists a coordinate system for
$M$ in which the points of $S$ are characterised by $x^1 = x^2 =
\cdots = x^{n-m} = 0$.

Solutions of differential equations are usually relations, say,
$\set{y_i = f_i(x^1, \dots, x^m), i=1, \dots, p}$, which can be
thought of as submanifolds with the coordinates $\set{x^1, \dots,
  x^m}$ of a larger manifold with coordinates $\set{y^1, \dots, y^p,
  x^1, \dots, x^m}$. Suppose $P$ is a point on a submanifold $S$
(dimension $m$) of $M$ (dimension $n$). A tangent space of a point in
$S$ is a subspace of the tangent space of the same point in $M$.

\section{Frobenius' Theorem}
\label{sec:frobenius-theorem}

In any coordinate patch $S$ there are coordinates $\set{y^a}$ and
basis vectors $\set{\pdv{y^a}}$ for vector fields on $S$; these basis
fields naturally commute. Any two fields on $S$ will have a Lie
Bracket which is tangent to $S$, and if a set of $m$ $C^{\infty}$
vector fields defined in a region $U$ of $M$ have Lie Brackets with
one another all of which are linear combinations of $m$ vector fields,
then the integral curves of the fields mesh to form a family of
submanifolds. Each point of $U$ is on one and only one submanifold,
and so the submanifolds fill $U$ in the same way as a congruence of
curves, and forms a \emph{foliation} of $U$, with each submanifold a
\emph{leaf}.

\begin{example}[The Generators of $S^2$]
  Consider the $\phi$-direction basis vector in spherical polars,
  \[\vec{e}_{\phi} = - y \vec{e}_z + x \vec{e}_y = \pdv{\phi} = -y
  \pdv{x} + x \pdv{y}\] In quantum mechanics we can define an
  operator, $\Op{l}_z \propto \pdv{\phi}$, defining $\Op{l}_x$, and
  $\Op{l}_y$ in similar ways, then there are commutation relations,
  \begin{align*}
    \comm{\Op{l}_x}{\Op{l}_y} &= - \Op{l}_z \\
    \comm{\Op{l}_y}{\Op{l}_z} &= - \Op{l}_x \\
    \comm{\Op{l}_z}{\Op{l}_x} &= - \Op{l}_y
  \end{align*}
  these generate a submanifold, apparently with three dimensions, but
  noting that each of the operators is tangent to a sphere of constant
  radius, and recalling that the contraction of this sphere with an
  operator is the number of spherical surfaces which the operator
  pierces, it's clear that all of the operators are tangent to the
  sphere, and so they generate a dimension 2 submanifold, the sphere,
  since they are linearly dependent.
\end{example}

\section{Invariance}
\label{sec:invariance}

A principle use of Lie derivatives is to express the notion of a
vector field's invariance under a transformation. A tensor field
$\ten{T}$ is invariant under a vector field $\vec{V}$ if
\[ \ld{\vec{V}}{\ten{T}} = 0 \] For example, if a system is invariant
under rotations in a given plane it is axisymmetric about that plane's
axis, and angular momentum is conserved.

Suppose we have a set, $F = \set{\ten{T_1}, \ten{T_2}, \dots}$ of
tensor fields whose invariance properties have been studied, then the
set of all vector fields $\vec{V}$ under which the fields of $F$ are
invariant form a Lie algebra. In the example of the angular momentum
operators, which are linearly dependent as fields on $R^3$, to
represent one field as a combination of the other two the linear
combination needs variable coefficients, so the fields are linearly
independent elements in the Lie Algebra, where the coefficients must
be constant.

\section{Killing Vector Fields}
\label{sec:kill-vect-fields}

A Killing vector field is a vector field $\vec{V}$ such that
\begin{equation}
  \label{eq:26}
  \ld{\vec{V}}{\ten{g}} = 0
\end{equation}
for $\ten{g}$ the metric tensor, which, in component notation
\begin{equation}
  \label{eq:27}
  \ten{\qty( \ld{\vec{V}}{g} )}_{ij} = V^k \pdv{x^k} \ten{g}_{ij}
                                     +\ten{g}_{ik} \pdv{x^j} V^k
                                     +\ten{g}_{kj} \pdv{x^i} V^k = 0
\end{equation}
The Killing vector field is then the metric which is invariant given a
specific vector field.  Using a coordinate system where the integral
curves of $\vec{V}$ are one family of coordinate lines, e.g.\~ for the
$x^1$ coordinate, then
\begin{equation}
  \label{eq:28}
  \ten{ \qty( \ld{\vec{V}}{g} )}_{ij} =\pdv{x^1} \ten{g}_{ij} =0
\end{equation}
So the metric components are independent of the coordinate $x^1$, and
conversely if there is a coordiante system where the representation of
the metric is independent of a certain coordinate the corresponding
basis vector to the coordinate is a Killing vector.

The metric tensor in Cartesian coordinates is $\ten{g}_{ij} =
\ten{\delta}_{ij}$ which is independent of $x$, $y$, and $z$, so the
Killing vectors are $\pdv{x}$, $\pdv{y}$, and $\pdv{z}$. In spherical
polar coordinates the same metric has the components $\ten{g}_{rr}=1$,
$\ten{g}_{\theta \theta}=r^2$, and $\ten{g}_{\phi \phi} = r^2
\sin[2](\theta)$.

Consider a system which is axisymmetric, with angle $\phi$, or is
close to axisymmetric (i.e.\~with a small perturbation from
axisymmetric), then, for an operator $\Op{L}$ and unknown $\psi$,
\begin{equation}
  \label{eq:29}
  \Op{L}(\psi) = 0
\end{equation}
where $\Op{L}$ is independent of a coordinate transformation $\phi \to
\phi + \textrm{const}$. The solutions of \eqref{eq:29} are not
necessarily axisymmetric, but scalar solutions can be Fourier-analysed
in $\phi$ as
\begin{equation}
  \label{eq:30}
  \psi(\phi, x^i) = \sum_{m=-\infty}^{\infty} \psi_m(x^j) e^{i m \phi}
\end{equation}
the functions $\psi_m(x^j)$ satisfy the related differential equation
\begin{equation}
  \label{eq:31}
  0 = \Op{L}_m(\phi_m) = e^{-i m \phi} L(\psi_m e^{im\phi})
\end{equation}
A solution $\psi$, is an axial eigenvalue, $m$, if
\begin{equation}
  \label{eq:32}
  \ld{\vec{e}_{\phi}} \psi = i m \psi
\end{equation}
for $\vec{e}_{\phi}$ tangent to the circles of symmetry. Any vector
field satisfying 
\[ \ld{\vec{e}_{\phi}}{\vec{V}} = i m \vec{V} \] can be expressed in
terms of a linear combination of vector axial harmonics with
eigenvalue $m$, of the form $\vec{e}_j e^{i m \phi}$, with
coefficients independent of $\phi$.

\section{Abstract Lie Groups}
\label{sec:abstract-lie-groups}

A Lie group is a differential manifold which has a differentiable
structure compatible with the group structure, that is, the operation
$G \times G \to G$ by $(x,y) \to x y^{-1}$ is a differentiable mapping.

Consider a finite-dimensional Lie group, $G$. Any neighbourhood of $e$
is mapped to a neighbourhood of $g$ by a mapping, which also carries
all of the tangent vectors, so the mapping is denoted $L_g : T_e \to
T_g$.\\
A vector $V$ is left-invariant if $L_g$ maps $V$ at $e$ to $V$ at $g$,
i.e. $L_g: V(e) \to V(g)$, for all $g$. It follows that $L_g$ maps
$V(h) \to V(gh)$ for any $h$ in $G$, giving a definition of a constant
vector field on $G$. If any two vector fields $V$ and $W$ are
left-invariant then $\comm{V}{W}$ is also a left-invariant field, so
the fields form a Lie algebra, $\mathfrak{L}(G)$, or $\mathfrak{g}$.

Let $\set{V_{(i)}}$ be a set of basis fields for a Lie algebra, then
\begin{equation}
  \label{eq:25}
  \comm{V_{(k)}}{V_{(l)}} = \ten{c}_{kl}^j V_{(j)}
\end{equation}
these $c$ are the structure constants which characterise the algebra;
if these disappear then the algebra is said to be Abelian. These
components transform as the components of a (1,2)-tensor, and each Lie
group and algebra has a unique structure tensor $\ten{C}$.

For an integral curve of a field $V$ which passes through $e$; it has
a tangent vector $V_e$, and a unique parameter $t$ for which $e$
corresponds to $t=0$. The points on the curve can be found by
eponentiation of $V$, $\exp(tV)$, defining a one-parameter subgroup of
$G$, and since each of these passes through $e$ there must be an
injective relationship between the one-parameter subgroups of $G$ and
its Lie algebra.

A theorem exists that states that every Lie algebra is the algebra of
one, and only one simply-connected Lie group.

\section{Group Representations} 
\label{sec:group-repr}

\newcommand{\group}[1]{\mathrm{#1}}

A group representation is a means of describing an abstract group in
terms of linear transformations of vector spaces.
\begin{definition}[Group Representation]
  A representation, $\rho$, of a group $\mathrm{G}$ on a vector space
  $V$ over a field $K$ is a group homomorphism from $\mathrm{G}$ to
  the general linear group, $\mathrm{GL}(V)$, \[ \rho : \mathrm{G} \to
  \mathrm{GL}(V) \] with the property \[ \rho(g_1 g_2) = \rho(g_1)
  \rho(g_2) \] for $g_1, g_2$ elements of $\mathrm{G}$.
\end{definition}
\begin{definition}[Faithful representation]
  A faithful representation occurs if the group homomorphism is an
  injective mapping, i.e. every element in the group is represented in
  the general linear group.
\end{definition}
\begin{example}
  Consider the cyclic group, $\group{C_3}=\set{1, u, u^2}$. This has a
  representation on $\mathbb{C}^2$ as
\[ \rho(1) = \begin{bmatrix}1&0\\0&1\end{bmatrix}, \quad \rho(u) =
\begin{bmatrix}
  1 & 0 \\ 0 & u
\end{bmatrix} \quad
\rho(u^2) =
\begin{bmatrix}
  1 & 0 \\ 0 & u^2
\end{bmatrix}
\]
with $u = e^{2 \pi i / 3}$, or, an isomorphic representation is
\[ \rho(1) = \begin{bmatrix}1&0\\0&1\end{bmatrix}, \quad \rho(u) =
\begin{bmatrix}
  u & 0 \\ 0 & 1
\end{bmatrix} \quad
\rho(u^2) =
\begin{bmatrix}
  u^2 & 0 \\ 0 & 1
\end{bmatrix}
\]
\end{example}

%%% Local Variables: 
%%% mode: latex
%%% TeX-master: "../project"
%%% End: 
