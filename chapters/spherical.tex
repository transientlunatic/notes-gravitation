\section{Orthogonal Metrics}
\label{sec:orthogonal-metrics}

An orthogonal metric takes the form
\begin{equation}
  \label{eq:78}
  \tensor{g}{_{\alpha \beta}} = 0 \qquad\forall\alpha \neq \beta
\end{equation}
implying that there are no cross-terms in the invariant interval. Thus
\begin{equation}
  \label{eq:79}
  \dd{s}^2 = \tensor{g}{_{\alpha \alpha}} (\dd{x}^{\alpha})^2
\end{equation}
The components of a metric will not be orthogonal in every coordinate
system; suppose that $\tensor{g}{_{\alpha \beta}}$ are the metric
components in a particular coordinate system where the metric is
orthogonal. Let $\tensor{g'}{_{\mu \nu}}$ be the components of the
metric in another coordinate system, then
\newcommand{\notimplies}{%
  \mathrel{{\ooalign{\hidewidth$\not\phantom{=}$\hidewidth\cr$\implies$}}}}
\begin{subequations}
\begin{align}
  \label{eq:80}
  \tensor{g'}{_{\mu \nu}} &= \pdv{x^{\alpha}}{x'^{\mu}} \pdv{x^{\beta}}{x'^{\nu}} \tensor{g}{_{\alpha \beta}} \\
&= \pdv{x^{\alpha}}{x'^{\mu}} \pdv{x^{\alpha}}{x'^{\nu}} \tensor{g}{_{\alpha \alpha}} \\
\notimplies \tensor{g'}{_{\mu \nu}} &= 0 \qquad \forall \mu' \neq \nu' \nonumber
\end{align}
\end{subequations}
The orthogonal metric components are closely related to the question
of whether the coordinate system has orthogonal basis vectors.  If we
have a coordinate system with basis vectors $\set{\vec{e}_i}$, and two
vectors $\vec{A} = A^i \vec{e}_i$, $\vec{B} = B^i \vec{e}_i$, then
\[ \vec{A} \vdot \vec{B} = (A^i \vec{e}_i) \vdot (B^j \vec{e}_j) = A^i B^j (\vec{e}_i \vdot \vec{e}_j) \]
so it follows
\[ \tensor{g}{_{ij}} = \vec{e}_i \vdot \vec{e}_j \] and
$\tensor{g}{_{ij}} = 0$ iff $\vec{e}_i$ and $\vec{e}_j$ are
orthogonal.

We normally want to choose a coordinate system in which the metric
coefficients are orthogonal, to simplify the expressions for
geometrical objects; if the contravariant metric components are
orthogonal then the diagonal terms are simply the reciprocal of the
covariant diagonal terms. To see this, we know
\begin{equation*}
  g^{\gamma \beta} g_{\gamma\gamma} = \delta^{\beta}_{\gamma} \quad\therefore \quad  g^{\gamma\gamma} = \frac{1}{g_{\gamma \gamma}}
\end{equation*}
The Christoffel symbols also take a simple form in an orthogonal
metric,
\begin{subequations}
  \begin{align}
    \label{eq:109}
    \Gamma^{\lambda}_{\mu \nu} &= 0 \qquad \text{for } \lambda, \mu, \nu \text{ all different.} \\
    \Gamma^{\lambda}_{\lambda \mu} = \Gamma^{\lambda}_{\mu \lambda} &= \quad \frac{g_{\lambda\lambda , \lambda}}{2 g_{\lambda \lambda}}\\
    \Gamma^{\lambda}_{\mu \mu} &= - \frac{g_{\mu \mu, \lambda}}{2 g_{\lambda \lambda}} \\
    \Gamma^{\lambda}_{\lambda \lambda} &= \quad \frac{g_{\lambda
        \lambda, \lambda}}{2 g_{\lambda \lambda}}
  \end{align}
\end{subequations}

For an affine parameter $s$ the geodesic equation takes the form
\begin{equation*}
  \dv{s} \qty( g_{\lambda \nu} \dv{x^{\nu}}{s} ) - \half \pdv{g_{\mu \nu}}{x^{\lambda}} \dv{x^{\mu}}{s} \dv{x^{\nu}}{s} = 0
\end{equation*}
for an orthogonal metric this reduces to
\begin{equation}
  \label{eq:112}
  \dv{s} \qty( g_{\lambda \lambda} \dv{x^{\lambda}}{s} ) - \half 
  \pdv{g_{\mu \mu}}{x^{\lambda}} \qty( \dv{x^{\mu}}{s} )^2 = 0
\end{equation}

\section{Spherically symmetric metrics}
\label{sec:spher-symm-matr}

Spherically symmetric solutions to the field equations are suitable
for describing the spacetime inside and around stars. In flat
Minkowski spacetime a polar coordinate system can be used to give an
invariant interval
\begin{equation}
  \label{eq:115}
  \dd{s}^2 = - \dd{t}^2 + \dd{r}^2 + r^2 \qty( \dd{\theta}^2 + \sin^2{\theta} \dd{\phi}^2 )
\end{equation}
Surfaces of constant $r$ and $t$ have the geometry of a 2-sphere with
an interval
\begin{equation}
  \label{eq:116}
  \dd{\ell}^2 = r^2 \qty( \dd{\theta}^2 + \sin^2{\theta} \dd{\phi}^2 )
\end{equation}
Thus a spacetime is spherically symmetric if every point in the
spacetime lies on a 2D surface which is a 2-sphere. Labelling the
coordinates $(r', t, \theta, \phi)$ then every point in a spherically
symmetric spacetime lies on a 2D surface, given by
\begin{equation}
  \label{eq:117}
  \dd{\ell}^2 = f(r', t) \qty[ \dd{\theta}^2 + \sin^2{\theta} \dd{\phi}^2]
\end{equation}
with $\sqrt{f}$ the radius of curvature of the 2-sphere.

In curved spacetime there is no trivial relation between the angular
coordinates of the two-sphere and the remaining coordinates at each
point in spacetime, but if we define
\begin{equation}
  \label{eq:118}
  r^2  = f(r', t)
\end{equation}
and we can line-up the origins of the 2-sphere coordinate systems
$(\theta, \phi)$ for points in spacetime with different values of
$r$. Spherical symmetry requires that any radial path in the space is
orthogonal to the 2D spheres on which the points along the radial path
lie, implying
\begin{equation}
  \label{eq:119}
  g_{r \theta} = g_{r \phi} = 0
\end{equation}
So the spacetime metric is reduced to the form
\begin{equation}
  \label{eq:120}
  \begin{split}
    \dd{s^2} = g_{tt} \dd{t^2} + 2 g_{tr} \dd{r}\dd{t} + 2 g_{t
      \theta} \dd{\theta}\dd{t} \\+ g_{rr} \dd{r^2} + r^2
    \qty(\dd{\theta^2} + \sin[2](\theta) \dd{\phi^2})
  \end{split}
\end{equation}
Considering the curve with $r$, $\theta$, and $\phi$ are constant,
which is a worldline of a particle in spacetime with constant spatial
coordinates; this curve must also be orthogonal to 2-spheres on which
each point lies, otherwise there would be a preferred direction in
spacetime. Thus
\begin{equation}
  \label{eq:121}
  g_{t\theta} = g_{t \phi} = 0
\end{equation}
so the general form for a metric in a spherically symmetric spacetime
is
\begin{equation}
  \label{eq:122}
  \begin{split}
    \dd{s^2} = g_{tt} \dd{t^2} + 2 g_{tr} \dd{r}\dd{t} + g_{rr}
    \dd{r^2} \\+ r^2 \qty( \dd{\theta^2} + \sin[2](\theta)
    \dd{\phi^2})
  \end{split}
\end{equation}
For $g_{tt}$, $g_{tr}$, and $g_{rr}$ arbitrary functions of $r$ and
$t$.

\section{Static Spacetime}
\label{sec:static-spacetime}

In s static spherically symmetric spacetime we can find a time
coordinate $t$ where
\begin{enumerate}
\item all metric components are independent of $t$
\item the metric is unchanged under a time-reversal operation, $t \to
  -t$.
\end{enumerate}
The second property implies $g_{tr}=0$, meaning that the interval is
\begin{equation}
  \label{eq:123}
  \dd{s^2} = - e^{\nu} \dd{t^2} + e^{\lambda} \dd{r^2} + r^2 \qty(\dd{\theta^2} + \sin[2](\theta) \dd{\phi^2})
\end{equation}
which is orthogonal. The functions $\nu(r)$ and $\lambda(r)$ replace
$g_{tt}$ and $g_{rr}$,since the exponential function is strictly
positive for all $r$, this is legitimate, provided $g_{tt}<0$ and
$g_{rr}>0$.

The Christoffel symbols for this spacetime are
\begin{subequations}
  \begin{align*}
    \label{eq:124}
    \Gamma^t_{rt}=\Gamma^t_{tr} &= \half \nu' & \Gamma^{\theta}_{r \theta} = \Gamma^{\theta}_{\theta r} &= \frac{1}{r} \\
    \Gamma^r_{tt} &= \half \nu' e^{\nu-\lambda} & \Gamma^{\theta}_{\phi \phi} &= - \sin(\theta) \cos(\theta) \\
    \Gamma^r_{rr} &= \half \lambda' & \Gamma^{\phi}_{r \phi} = \Gamma^{\phi}_{\phi r} &= \frac{1}{r} \\
    \Gamma^r_{\theta \theta} &= - r e^{- \lambda} &
    \Gamma^{\phi}_{\theta \phi} = \Gamma^{\phi}_{\phi \theta} &=
    \cot(\theta)
  \end{align*}
  \begin{equation*}
    \Gamma^r_{\phi\phi}  = -r e^{-\lambda} \sin[2](\theta)
  \end{equation*}
\end{subequations}

The Ricci tensor is given by
\begin{equation}
  \label{eq:125}
  R_{\lambda \nu} = \Gamma^{\tau}_{\lambda \nu} \Gamma^{\sigma}_{\tau\sigma} - \Gamma^{\tau}_{\lambda \sigma} \Gamma^{\sigma}_{\tau\nu} + \Gamma^{\sigma}_{\lambda\nu,\sigma} - \Gamma^{\sigma}_{\lambda\sigma,\nu}
\end{equation}
so
\begin{subequations}
  \begin{align}
    \label{eq:126}
    R_{tt} &= \half e^{\nu-\lambda} \qty( \nu'' + \half \nu'^2 - \half \nu' \lambda' + \frac{2}{r} \nu') \\
    \label{eq:129}
    R_{rr} &= - \half \qty( \nu'' + \half \nu'^2 - \half \nu' \lambda' - \frac{2}{r} \lambda') \\
    \label{eq:130}
    R_{\theta\theta} &= 1 - e^{-\lambda} \qty[1+\frac{r}{2}(\nu'-\lambda')] \\
    \label{eq:131}
    R_{\phi\phi}&= R_{\theta\theta} \sin[2](\theta)
  \end{align}
\end{subequations}

\section{The Schwarzschild metric}
\label{sec:schwarzschild-metric}

We can derive the metric for the spacetime exterior to a star from the
static spherically symmetric metric; the Schwarzchild metric; if the
star is in an isolated region of space we can assume all components of
the Ricci tensor to be zero, so
\begin{equation}
  \label{eq:127}
  e^{\lambda-\nu}R_{tt}+R_{rr} = \frac{\nu'+\lambda'}{r}=0
\end{equation}
which implies that $\nu+\lambda$ is constant. At a large distance from
the star the metric should reduce to special relativity, so as
\[ r \to \infty, \quad e^{\nu}\to 1, e^{\lambda} \to 1 \implies \nu \to 0, \lambda \to 0\]
and so $\nu+\lambda=0$, giving
\begin{equation}
  \label{eq:128}
  e^{\nu} = e^{-\lambda}
\end{equation}
This lets us eliminate $\nu$ from the $R_{\theta\theta}$ expression,
equation \eqref{eq:130}, so
\begin{equation}
  \label{eq:132}
  e^{-\lambda} (1-\lambda'r) = 1 \implies \dv{r} \qty(r e^{-\lambda})=1
\end{equation}
Integrating this we get
\begin{equation}
  \label{eq:133}
  e^{\nu} = e^{-\lambda} = 1 + \frac{\alpha}{r}
\end{equation}
where $\alpha$ is a constant.

Consider a material test particle, with so little rest mass that it
does not disturb the metric, which is released from rest, then
\begin{equation}
  \label{eq:134}
  \dd{x^j}{\tau} = 0 \quad j = 1,2,3
\end{equation}
for $\tau$ the proper time, and
\begin{equation}
  \label{eq:135}
  \dv{x^0}{\tau} \equiv \dv{t}{\tau} \neq 0
\end{equation}
Realling that 
\[ g_{\alpha\beta} \dv{x^{\alpha}}{\tau} \dv{x^{\beta}}{\tau} = -1 \]
then
\begin{equation}
  \label{eq:136}
  \dv{t}{\tau} = e^{- \frac{\nu}{2}}
\end{equation}
Applying the geodesic equations, equation \eqref{eq:56}, at the
instance that the particle is released this reduces to
\begin{equation}
  \label{eq:137}
  \dv[2]{r}{\tau} + \Gamma_{tt}^r \qty( \dv{t}{\tau})^2 =0
\end{equation}
Substituting the Christoffel symbol and equation \eqref{eq:136} we
obtain
\begin{equation}
  \label{eq:138}
  \dv[2]{r}{t} = \frac{\alpha}{2 r^2}
\end{equation}
In the limit of a weak field this reduces to Newtonian gravity,
\begin{equation}
  \label{eq:139}
  \dv[2]{r}{t} = - \frac{GM}{r^2}
\end{equation}
for $M$ the mass of the star, meaning $\alpha = -2GM =-2M$ for
$G=1$. Thus the invariant interval is
\begin{equation}
  \label{eq:140}
\begin{split}
  \dd{s^2} = - \qty(1-\frac{2M}{r}) \dd{t^2} + \frac{\dd{r^2}}{\qty(1-\frac{2M}{r})} \\
+ r^2 \dd{\theta^2} + r^2 \sin[2](\theta) \dd{\phi^2}
\end{split}
\end{equation}

%%% Local Variables: 
%%% mode: latex
%%% TeX-master: "../project"
%%% End: 
