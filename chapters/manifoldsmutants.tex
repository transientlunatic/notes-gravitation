
\section{Manifolds}
\label{sec:manifolds}

$\rn$ is the set of all $n$-tuples of real numbers. A set $M$ is
defined to be a manifold if each element of $M$ has an open
neighbourhood with a continuous bijection onto an open set of $\rn$
for some $n$. This mathematical object is rather degenerate, and
hasn't got, for example, a notion of distance, and what structure is
defined is only at a local level.

The elements, $P$ of $M$ have a unique $n$-tuple, the co-ordinates of
$P$:
\begin{equation}
  \label{eq:5}
  \qty( x_1(P), x_2(P), \dots, x_n(P) )
\end{equation}

\begin{figure}[b]
  \centering
  \begin{tikzpicture}[scale=0.45]
\begin{scope}[xshift=-50]
	\shade[left color=muted-blue!10,right color=muted-blue!80] 
	  (0,0) to[out=-10,in=150] (6,-2) -- (12,1) to[out=150,in=-10] (5.5,3.7) -- cycle;

	\begin{scope}[scale=0.4, xshift=300]
		\coordinate (A) at (0,0);
		\draw node at (6, -0.5) {$U$};
		\fill[opacity=0.1, fill=black, dotted] 
		  (0,0) to[out=-10,in=150] (6,-2) -- (12,1) to[out=150,in=-10] (5.5,3.7) -- cycle;
	\end{scope}
	\begin{scope}[scale=0.3, xshift=400, yshift=100]
		\coordinate (B) at (12,1);
		\draw node at (6, 2.5) {$V$};
		\fill[opacity=0.1, fill=black, dotted] 
		  (0,0) to[out=-10,in=150] (6,-2) -- (12,1) to[out=150,in=-10] (5.5,3.7) -- cycle;
	\end{scope}
\end{scope}

\begin{scope}[yshift=-250, xshift=-50]
	\coordinate (C) at (0, 2.5);
	\draw [->] (0,0) -- (5,0); \draw [->] (0,0) -- (0,5); \fill [opacity=0.1] (0,0) rectangle (5,5);
	\fill [opacity=0.1] (3,3) rectangle (5,5);
	\draw (2.5,.5) node {$f_U(u)$};
\end{scope}
\begin{scope}[yshift=-250, xshift=200]
	\coordinate (D) at (5,2.5);
	\coordinate (E) at (1,1);
	\draw [->] (0,0) -- (5,0); \draw [->] (0,0) -- (0,5); \fill [opacity=0.1] (0,0) rectangle (5,5);
	\fill [opacity=0.1] (0,0) rectangle (2,2);
	\draw (2.5,4.5) node {$f_V(v)$};
\end{scope}

\draw [ultra thick, accent-red]  (A) edge [out=180, in=180, ->] node [black, fill=white, pos=0.3]  {$f_U$} (C) ;
\draw [ultra thick, accent-blue]  (B) edge [out=0, in=0, ->] node [black, fill=white, pos=0.4]  {$f_V$} (D);

\draw [ultra thick, accent-green] (C)+(4,1.5) edge [out=0, in=180,->] node [black, fill=white, pos=0.5]  {$f_V \circ f^{-1}_U$} (E);
\end{tikzpicture}
%%% Local Variables: 
%%% mode: latex
%%% TeX-master: "../project"
%%% End: 

  \caption{Open sets $U$ and $V$ of a manifold which can be mapped, by the functions $f_U$ and $f_V$ onto open sets in $\mathbb{R}^n$, $f_U(u)$ and $f_V(v)$.}
  \label{fig:manifolds}
\end{figure}

Let $f:U(P) \to \rn$ be a function from a neighbourhood $U$ around
$P\in M$ of $M$ onto an open set $f(U) \in \rn$, then the pair,
$(U,f)$ are a chart. The open neighbourhoods must overlap if all
points of $M$ must be included in at least one, and so these overlaps
give an additional characterisation of the manifold. Suppose $V$ is a
neighbourhood overlapping $U$, such that $V$ has a map $g$ onto an
open region of $\rn$, this region may be completely disjoint from the
equivalent region for $U$. The open intersection of $U$ and $V$ is
described by two coordinate systems, so there should be an expression
connecting the two. Such an equation is a \emph{coordinate
  transformation}.

\begin{definition}[Manifold]
  A manifold is a second-countable Hausdorff space which is locally
  homeomorphic to Euclidean space.
\end{definition}
\begin{definition}[Chart]
  A chart $(U, f_U)$ of a manifold is an open set $U$ together with a
  mapping $f_U: U \to \rn$.
\end{definition}
\begin{definition}[Coordinates]
  The coordinates $(x^1, \cdots, x^n)$ of the image $f_U(x) \in \rn$
  are the coordinates of $x$ in the chart $(U, f_U)$; the local
  coordinate system.
\end{definition}

If a whole system of charts, an atlas, can be constructed such that
every point in $M$ is in at least one neighbourhood, and every chart
is $C^k$-related to every overlapping neighbour the manifold is
$C^k$. Any such manifold for $C^1$ or greater is a differentiable
manifold.

Many mathematical spaces can be considered as manifolds, for example
the space of Lorentz boosts, mechanical phase space, solution spaces
for functions, and vector spaces. Groups also have representations as
manifolds; a Lie group, e.g. the group of rotations of a rigid object
in space is also a $C^{\infty}$ manifold.

\begin{figure}[b]
  \centering
  
\newcommand{\mobstrip}[1]{%
\draw [very thick, top color=white,
       bottom color=muted-blue, rotate=#1,
       draw=muted-blue
      ]
 (0:2.8453)    ++ (-30:1.5359) arc (60:0:2)
            -- ++  (90:5) arc (0:60:2) 
            -- ++ (150:3) arc (60:120:2) 
            -- ++ (210:5) arc (120:60:2) -- cycle;
}%

\begin{tikzpicture} [rotate=22, scale=0.5]
   \begin{scope} [transform shape]
     \mobstrip{0}
     \mobstrip{120}
     \mobstrip{-120}
     \draw [
     ultra thick, accent-red,
     decoration={markings, mark=at position 0.625 with {\arrow{>}}},
        postaction={decorate}
     ] (-60:3.5) circle (1);
     \draw [
     ultra thick, accent-red,
          decoration={markings, mark=at position 0.625 with {\arrow{<}}},
        postaction={decorate}
     ](180:3.5) circle (1);
  %   % redraw the first strip after clipping
  %   \clip (-1.4,2.4)--(-.3,6.1)--(1.3,6.1)--(5.3,3.7)--(5.3,-2.7)--cycle;
  %   \strip{0}
  %   \draw (60:3.5) node [gray,xscale=-4,yscale=4,rotate=30]{b};
   \end{scope}
\end{tikzpicture}
%%% Local Variables: 
%%% mode: latex
%%% TeX-master: "../project"
%%% End: 

  \caption{The M\"obius strip, a non-orientable manifold; as such,
    there is no consistent notion of a clockwise or an anti-clockwise
    orientation across the whole manifold.}
  \label{fig:mobius}
\end{figure}

\begin{definition}[Orientability]
  Let $M^n$ be a differentiable $n$-dimensional manifold equipped with
  an atlas $\set{\phi_a}$, and suppose that for any two charts of the
  atlas, $\phi_{\alpha}$, and $\phi_{\beta}$ the Jacobian of the
  transition function $\phi_{\beta \alpha} = \phi_{\beta} \circ
  \phi_{\alpha}^{-1}$ is positive at all points in the domain, then
  $M$ is an orientable manifold.
\end{definition}
A Jacobian which is negative will reverse the orientation of a
coordinate system.

\section{Global Properties}
\label{sec:global-properties}

At a local level all manifolds look like $\rn$, but on a global level
this is not the case. For example a sphere, $S^2$ and a crayon are
both two-dimensional manifolds, and neither has a single map to
$\mathbb{R}^2$, but there is a continuous map from one to the other.
If there is a $C^{\infty}$ bijective map from one $C^{\infty}$
manifold to another, then the two map is a \emph{diffeomorphism} from
one manifold to the other.

\begin{figure}[t]
  \centering
  \begin{tikzpicture}[scale=0.45]
    \begin{scope}[yshift=120]
      \draw[ultra thick, muted-green] (0,0) -- (10,0); 
      \node at (-1.2,0) {$\lambda \in \mathbb{R}^1$} ;
      \foreach \coor in {0,1,...,10}
      \coordinate (re\coor) at (\coor, 0);
    \end{scope}
    \begin{scope}[xshift=-50]
	\shade[left color=muted-blue!10,right color=muted-blue!80] 
	  (0,0) to[out=-10,in=150] (6,-2) -- (12,1) to[out=150,in=-10] (5.5,3.7) -- cycle;

	\begin{scope}[scale=0.4, xshift=300]
		\coordinate (A) at (0,0);
		\draw node at (11.5, 0.8) {$U$};
		\fill[opacity=0.1, fill=black, dotted] 
		  (0,0) to[out=-10,in=150] (6,-2) -- (12,1) to[out=150,in=-10] (5.5,3.7) -- cycle;
	\end{scope}

\draw[muted-orange,line width=1.5pt,shorten >= 3pt,shorten <= 3pt] 
  (0,0) .. controls (6,-0.2) and (5,1.5) ..
  coordinate[pos=0.0] (cux0)
  coordinate[pos=0.1] (cux1) 
  coordinate[pos=0.2] (cux2) 
  coordinate[pos=0.3] (cux3) 
  coordinate[pos=0.4] (cux4) 
  coordinate[pos=0.5] (cux5) 
  coordinate[pos=0.6] (cux6) 
  coordinate[pos=0.7] (cux7) 
  coordinate[pos=0.8] (cux8) 
  coordinate[pos=0.9] (cux9) 
  coordinate[pos=1.0] (cux10) (12,1);
\draw (cux10)+(0.3,0) node [right] {$P \in M$};

% we draw the filled red circles on the red line
\foreach \coor in {0,1,...,10}{
  \draw (cux\coor) edge  [accent-blue, shorten >=0.1cm, 
                          shorten <= 0.1cm, line width=0.3mm, 
                          <-, out=90, in=270] 
        (re\coor);
  \draw[fill=muted-green, draw=white] (re\coor) circle (3pt);
  \draw[fill=muted-orange, draw=white] (cux\coor) circle (3pt);
}
\draw (13, 3) node {$f$};
\draw (3, -2) node {$g$};
\end{scope}
\begin{scope}[yshift=-210, xshift=-50]
	\coordinate (C) at (0, 0.5);
	\draw [->] (0,0) -- (12.1,0); 
        \draw [->] (0,0) -- (0,5); 
        \fill [opacity=0.1] (0,0) rectangle (12,5);
        \draw (10,0.5) node {$g(U) \in \mathbb{R}^n$};
        \node (E) at (12, 4);

        \draw [ultra thick, accent-green, out=0, in=180] 
        (0, 0.5) edge % [out=0, in=180] 
          coordinate[pos=1/10] (cdx4)
          coordinate[pos=3/10] (cdx5)
          coordinate[pos=6/10] (cdx6)
          coordinate[pos=9/10] (cdx7)
        (8, 5);
\end{scope}

\foreach \coor in {4,5,...,7}{
 %\draw[fill=muted-green, draw=white] (re\coor) circle (3pt);
  \draw[fill=accent-green, draw=white] (cdx\coor) circle (3pt);
}

\foreach \coor in {4,5,...,7}{
   \draw (cux\coor) edge  [accent-red, shorten >=0.1cm, 
                           shorten <= 0.1cm, line width=0.3mm, 
                           ->, out=270, in=90] 
         (cdx\coor);
}


\end{tikzpicture}
\caption{A curve defined over a manifold.}
  \label{fig:manifold-curve}
\end{figure}

\section{Curves and Functions on Manifolds}
\label{sec:curves}

A curve is a differentiable mapping from an open set of $\mathbb{R}^1$
into $M$, so each point $\lambda$ on the real line is associated with
a point in $M$.

A function on $M$ is a rule assigning a real number to each point of
$M$. When a region of $M$ is mapped to $\rn$ the function becomes a
function of $\rn$. If this is differentiable on $\rn$ it is defined as
being differentiable on $M$. That is
\[ g(P \in M) = g(x_1(P), \dots, x_n(P)) = g(x_1, \dots x_n) \]

Now, consider a curve, $c$, which passes through $P \in M$, which is
described by $x^i = x^i(\lambda)$ for $i \in \set{1, \dots, n}$, and a
differentiable function $g(x^i)$ on $M$. $g$ has a value at each point
along the curve, so there is a differentiable function, $g(\lambda)$
along the curve which gives the value of $c$ at the point with a parameter value $\lambda$,
\[ g(\lambda) = g(x^1(\lambda), \dots, x^n(\lambda) ) =
g(x^i(\lambda)) \]
Differentiating, and applying the chain rule,
\begin{equation}
  \label{eq:6}
  \dv{g}{\lambda} = \dv{x^i}{\lambda} \dv{g}{x^i}
\end{equation}
using Einstein notation.

In Euclidean space the set $\set{ \dv*{x^i}{\lambda} }$ would be
components of a vector tangent to the curve $x_i(\lambda)$, and would
be the tangent to an infinite number of curves through $P$.

Manifolds do not necessarily have a notion of distance between points,
so vectors must be defined in terms only of infinitessimal
neighbourhoods of the points of $M$. 

\section{Vectors}
\label{sec:vectors}

Suppose $a$ and $b$ are two numbers and $x^i = x^i(\mu)$ is another curve through $P$, then at $P$,
\[ \dv{\mu} = \dv{x^i}{\mu} \pdv{x^i} \]
and
\[ a \dv{\lambda} + b\dv{\mu} = \qty( a \dv{x^i}{\lambda} + b \dv{x^i}{\mu} ) \pdv{x^i} \]
The numbers inside the bracket are the components of a new vector, tangent to a curve at $P$, so a curve must exist, with parameter $\phi$ such that at $P$,
\[ \dv{\phi} = \qty( a \dv{x^i}{\lambda} + b \dv{x^i}{\mu} ) \pdv{x^i} \]
collecting this, at $P$,
\[ a \dv{\lambda} + b \dv{\mu} = \dv{\phi} \]
forming a vector space.

In any given coordinate system there are special curves, the
coordinate lines. The derivations along them are $\pdv*{x^i}$, and
$\dv*{\lambda}$ has components on the basis, showing that
$\set{\dv*{x^i}{\lambda}}$ are a basis for a vector space.

\section{Basis Vectors and basis vector fields}
\label{sec:basis-vectors-basis}

At any point $P$ the space $T_P$ is a vector space with the same
dimension $n$ as the manifold, and is the tangent space. Any
collection of linearly independent vectors can form a basis, and if
there is a coordinate system $\set{x^i}$ in a neighbourhood $U$ of $P$
then these coordinate define a coordinate basis at all points in $U$.

\begin{definition}[Tangent Vector]
  Let $M^m$ be a differential manifold, $p$ a point on it, and $U$ be
  an open neighbourhood of $M$. Let $\epsilon>0$, and $\gamma :
  (-\epsilon, \epsilon) \to U$ be a differentiable curve on $M$ such
  that $\gamma(0) = p$. For any differentiable function $f:U \subset M
  \to \mathbb{R}$ the directional derivative of $f$ along $\gamma$ to
  be the number
  \[ D_{\gamma}(f) = \eval{\dv{t}( f (\gamma(t) ) )}_{t=0} \] The
  operator $D_{\gamma}$ is called the tangent vectorto $\gamma$ at
  $p$.
\end{definition}
\begin{definition}[Tangent Space]
  The tangent space of $M$ at $p$, $T_p$ is a vector space with the
  same dimension as the manifold composed of all of the tangent
  vectors to the manifold at $p$. If $x$ is a coordinate system which
  contains $p$, then the set of $\pdv*{x}$ are the basis of the tangent
  space; and this is a coordinate basis.
\end{definition}

At any point $P$ an arbitrary vector $\vec{V}$ is
\[ \vec{V} = V^i \pdv{x^i} = V^j \bar{e}_j = V^j \partial_j\] for an
arbitrary basis $\set{\vec{e}_i}$. Thus $\set{V^i}$ are the components
of $\vec{V}$ in $\set{ \pdv*{x^i}}$, and $\set{V^j}$ the components in
$\vec{e}_j$, where a useful notation, $\partial_i \equiv \pdv*{x^i}$ is
introduced.

\section{Differential of a map}
\label{sec:differential-map}

\begin{definition}[Differential]
  Let $\phi: M^m \to N^n$ be a differentiable map between
  differentiable manifolds. We define the differential of $\phi$ at $p
  \in M$ as the linear transformation between vector spaces
  \begin{align*}
    \dt{\phi_p} : T_pM & \to T_{\phi(p)N} \\
D_{\gamma} &\mapsto D_{\phi \circ \gamma}
  \end{align*}
\end{definition}

\section{Fibre bundles}
\label{sec:fiber-bundles}



\begin{figure}[b]
  \centering
  \begin{tikzpicture}
  \draw [thick, ->] (1,0) -- (7.5,0);
  \node at (8,0) {$B$};

  \draw [thick, ->] (0,0.5) -- (0,3);
  \node at (0.5, 3) {$\mathbb{R}^n$};

  \fill [opacity=0.1] (1,0.5) rectangle (7, 3);
  \node at (6.5,2.7) {$E$};
  \foreach \x in {3,3.5,...,6}{
    \draw (\x, 0.5) -- (\x, 3);
  }

  \draw (7, 1.5) edge [shorten >=0.1cm, shorten <=0.1cm, ->, 
                       accent-red, out=0, in=90, thick]
   node [midway, right] {$\pi$} (7.3,0);

   \draw [<->] (3,-0.25) -- (6,-0.25) node [midway, fill=white] {$U$};
   \draw (3,0) node {$($}; \draw (6,0) node {$)$};
   \filldraw [fill=white] (4.5,0) circle (0.07) node [above] {$x$};
   \draw (4.5, 3.5) node {fibre $\pi^{-1}(x)$} edge [thick, accent-blue, ->] (4.5, 3);
\end{tikzpicture}


%%% Local Variables: 
%%% mode: latex
%%% TeX-master: "../project"
%%% End: 

  \caption{A fibre bundle, $E$, over a base $B$, with a projection
    function $\pi$. Each fibre at a point $x \in B$ is then described
    by a function $\pi^{-1}(x)$, and the whole bundle over an open
    region $U$ is  $\pi^{-1}(U)$.}
  \label{fig:fibre-bundle}
\end{figure}

\begin{definition}[Homeomorphism]
A homeomorphism is an bijection from one space to the other which is
continuous, and its inverse is continuous.
\end{definition}

\begin{definition}[Fibre Bundle]
A fibre bundle is a space, $E$, which has a base manifold $B$, a
projection $\pi: E \to B$, a typical fibre $F$ a structure group $G$
of homeomorphisms of $F$ onto iteself, and a family $\set{U_j}$ of
open sets which cover $B$, which satisfy
\begin{enumerate}
\item Locally the bundle is trivial: the bundle over any set $U_j$
  which is just $\pi^{-1}(U_j)$ has a homeomorphism onto $U_j \times
  F$. Part of this is a homeomorphism from each fibre, say
  $\pi^{-1}(x)$, $x\in B$ onto $F$. Call this map $h_j(x)$ labelled
  both by $x$ defining the fibre, but also by $j$ indexing the set
  $U_j$ which contains $x$.
\item When two sets $U_j$ and $U_k$ overlap a point $x$ in their
  intersection has two homeomorphisms, $h_j(x)$ and $h_k(x)$ from the
  fibre onto $F$. Thus $h_j(x) \circ h_k^{-1}(x)$ is a homeomorphism
  of $F$ onto $F$. This must be an element of $G$.
\end{enumerate}
\end{definition}

In two overlapping neighbourhoods there are homeomorphisms $h_j(x)$
and $h_k(x)$ which map $\vec{v}$ to $\alpha_{(j)}$ and to
$\alpha_{(k)}$, and so the homeomorphism $h_j(x) \circ h_k^{-1}(x): F
\to F$ maps $\alpha_{(k)} \to \alpha_{(j)}$, which is equivalent to
multiplication by a real number $r_{jk} = \alpha_{(j)}/a_{(k)}$, which
must be a non-zero real number. Thus the structure group is
$\mathbb{R}^1 - \set{0}$, which is a Lie group under multiplication.

The difference between two bundles over a base space is the bundles'
\emph{structure group}.

If the coordinates $\lambda_j$ can be arranged in such a way that
$\lambda_j$ and $\lambda_k$ increase in the same direction in $S^1$ in
the overlap region, then $S^1$ is orientable. In such an overlap all
$r_{jk}$ are positive, so the structure group becomes $\mathbb{R}^+$,
and by scaling the coordinates so that $\dv*{\lambda_j}{\lambda_k} =
1$ in every overlap the group reduces to the identity element, thus
the structure group is trivial and we have a bundle structure.

Consider the combination of a manifold $M$ with all of its tangent
spaces $T_P$. This is equivalent to the set of all vectors at all
points in the manifold, and can be defined as a new manifold $TM$; a
\emph{fibre bundle}, with the fibres the spaces $T_P$, and has
dimension $m+n$, with $m$ the dimension of each fibre, and $n$ of the
base manifold. An example of a fibre bundle is the structure of space
and time in the Newtonian world view, where the base space is
$\mathbb{R}^1$, with $\mathbb{R}^3$ fibres, that is a base of time
with fibres of space. This follows, as there is no absolute concept of
space in Newtonian physics, and so there is no connection between
points on each fibre.

\begin{definition}[Cartesian Product Space]
A Cartesian product space of two spaces $M$ and $N$, denoted $M \times
N$ is the set of all ordered pairs $(a,b)$ for $a \in M$ and $b \in
N$, and if $M$ and $N$ are manifolds then $M \times N$ is also a
manifold; the set of coordinates $\set{x^i}$ of an open set $U$ of $M$
taken with $\set{y^i}$ of an open set $V$ of $N$ form a set of $m+n$
coordiates for the open set $(U,V)$ of $M \times N$.
\end{definition}

Fiber bundles are locally product spaces; they are locally trivial, as
they look like product spaces, but not usually globally trivial. For
example, $TS^2$ is the tangent bundle of a 2-sphere; were it globally
tivial there would be a diffeomorphism of $TS^2$ onto $S^2 \times
\mathbb{R}^2$. Consider the set $(P, \bar{V})$ in the product space;
the invserse map gives a nowhere-zero cross-section of $TS^2$, but a
vector field must always have a zero, and so $TS^2$ has no global
product structure. A tangent bundle $TS^1$ \emph{is} identical to $S^1
\times \mathbb{R}$, but if the circle is cut at a point $P$, and a
twist is introduced, making a M\"obius strip, the resulting bundle is
not a product space, as no bijection exists from one bundle onto all
of the other.

\begin{definition}[Tangent Bundle]
  Let $M^n$ be a differentiable manifold. The disjoint union of all
  tangent planes to $M$, \[ \coprod_{p \in M} T_p M \] is a vector
  bundle with the fibre $\rn$ over $M$, called the tangent bundle to
  $M$, denoted $TM$.
\end{definition}

\section{Integral Curves}
\label{sec:vector-fields}

For a $C^1$ vector field there is always a curve which has the field as a tangent at each of its points; these are its integral curves.

Let the components of the field be $V^i(P) = v^i(x^j)$ in a coordinate
system $\set{x^i}$ The statement that this is a tangent vector to a curve parameterised by $\lambda$ is
\begin{equation}
  \label{eq:7}
  \dv{x^i}{\lambda} = v^i(x^j)
\end{equation}
which is a set of first-order ODEs for $x^i(\lambda)$, which always
has a unique solution in a neighbourood of $P$.

\section{Lie Brackets}
\label{sec:lie-brackets}

Any linearly independent set of vector fields can serve as a basis,
but not all come from coordinate systems; all coordinate systems commute, but in general not all vector fields commute, and the commutator,
\[ \comm{V}{W} = V W - W V \] is a vector field with, in general,
non-vanishing components. If $W$ and $V$ are elements of a basis they
cannot be expressed as derivatives with respect to any
coordinates. The commutator is the Lie Bracket of the two vector
fields, and shows that the parameterisation of the integral curves of
these vector fields is not that of a coordinate system.

\begin{definition}[Lie Bracket]
  Let $X$ and $Y$ be vector fields of class $C^1$, then the Lie
  Bracket is a vector field, defined by the operation $f \mapsto (XY -
  YX) f$, and is denoted $\comm{X}{Y}$.
\end{definition}

The vector $\comm{V}{W}$ can be viusualised as the difference in
moving a distance $\Delta \lambda = \epsilon$ along the $V$ curve,
then $\Delta \mu = \epsilon$ along a $W$ curve, ending at a point
$A$. Going in the opposite order we end at a point $B$. The vector
between these two points is $\epsilon^2 \comm{V}{W}$.

A Lie Algebra of vector fields on a region $U$ of $M$ is a set $A$ of
vector fields on $U$ which is a vector space under addition, which is
closed under the Lie Bracket operation.

For a basis to be a coordinate basis the fields of the basis must commute,
\[\comm{A}{B} = 0 \]

\begin{figure}[t]
  \centering
  \begin{tikzpicture}[scale=0.7]

\begin{scope}[]
	\draw [thin, ->] (-.2, 0) -- (5.2,0);
	\draw [thin, ->] (0, -.2) -- (0,5.2); 

\clip (0,0) rectangle (5,5);
\draw [help lines] (0,0) grid (5,5);

	\draw [accent-red, ultra thick, ->] (1,1) -- (4,1) node [left, midway, below, black] {$\partial_\mu$};
	\draw [accent-red, ultra thick, ->] (1,4) -- (4,4) node [left, midway, above, black] {$\partial_\mu$};

	\draw [accent-blue, ultra thick, ->] (1,1) -- (1,4) node [left, midway, black] {$\partial_\nu$};
	\draw [accent-blue, ultra thick, ->] (4,1) -- (4,4) node [right, midway, black] {$\partial_\nu$};
\end{scope}

\begin{scope}[xshift=170]
 {

	}
	\draw [thin, ->] (-.2, 0) -- (5,0);
	\draw [thin, ->] (0, -.2) -- (0,5); 

	\draw [accent-red, ultra thick, ->] (1,1) -- (4,2) node [left, midway, below, black] {$W$};
	\draw [accent-red, ultra thick, ->] (1,4) -- (4,5) node [left, midway, above, black] {$W$};

	\draw [accent-blue, ultra thick, ->] (1,1) -- (1,4) node [left, midway, black] {$V$};
	\draw [accent-blue, ultra thick, ->] (4,2) -- (5,5) node [right, midway, black] {$V$};

\clip (0,0) rectangle (5,5);
	\clip (-.2,-.2) rectangle (5.5,5.2);
		\foreach \x in {0,0.55,...,4}{
		\draw [help lines] (\x-.06, 0) -- (\x+\x-1.2,6);
	}
	\foreach \y in {0.3,...,6}{
		\draw [help lines] (0,\y-0.6) -- (6, \y+1.3);
	}	



\end{scope}

\end{tikzpicture}
  \caption{A geometrical interpretation of the commutator. In the top diagram are two commuting vector fields, but in the lower diagram the two fields do not commute. Moving a distance $\epsilon$ along $V$ then $\epsilon$ along $W$ does not take us to the same place as moving $\epsilon$ along $W$ and then $\epsilon$ along $V$, so the paths do not meet, with a gap equal to $\epsilon^2\comm{V}{W}$ separating their end points.}
  \label{fig:commutator}
\end{figure}

\section{One-forms}
\label{sec:one-forms}

Consider $T_P$, the tangent space of vectors at a point $P$. A
one-form, $\of{\omega}$ at $P$ associates a vector $\vec{V}$ at P to a
real number, $\of{\omega}(\vec{V})$. 

\begin{subequations}
This function is linear; for vectors $\vec{V}$ and $\vec{W}$, and real numbers $a$ and $b$,
\begin{equation}
\label{eq:8}
 \of{\omega}( a \vec{V} + b \vec{W} ) = a \of{\omega}(\vec{V}) + b
\of{\omega}(\vec{W}) 
\end{equation}
can be multiplied by scalars,
\begin{equation}
  \label{eq:9}
  (a \of{\omega})(\vec{V}) = a [ \of{\omega}(\vec{V})]
\end{equation}
and have the property
\begin{equation}
  \label{eq:10}
  (\of{\omega} + \of{\sigma}) (\vec{V}) + \of{\omega}(\vec{V}) + \of{\sigma}(\vec{V})
\end{equation}
\end{subequations}
They satisfy the axioms of a vector space, and are indeed the duals of vectors, and have a tangent vector space $T^{*}_P$.

A field of one-forms can represent the gradient of a function $f$;
such a field is denoted $\dd{f}$,
\[ \of{\dd{}}f (\dv*{\lambda}) = \dv{f}{\lambda} \]

In the vector space $T^{*}_P$ any $n$ linearly independent one-forms
can constitute a basis, however selecting a set of basis vectors,
$\set{\vec{e}_i}$ in $T_P$ induces a preferred basis on $T^{*}_P$, the
dual basis $\set{\of{\omega}^i}$.

These have the property
\[ \of{\omega}^i(\vec{e}_j) = \delta^i_j \]

\section{Tensors}
\label{sec:tensors}

\newcommand{\tensororder}[2]{#1 \choose #2}

Consider a point $P$ in a manifold $M$, a \emph{tensor} of type
$\tensororder{N}{N'}$ at $P$ is a linear function with $N$ one-form
arguments, and $N'$ vector arguments, which has a real scalar
value. Tensors are linear on every argument.

A $\tensororder{N}{N'}$ tensor field is a rule which assigns a
$\tensororder{N}{N'}$ tensor to each point in a space, and are
functions on $M$.

It is possible to construct higher-order tensors out of lower order
ones with the tensor (outer) product operation; for two vectors $V$
and $W$ the tensor product is defined such that
\begin{equation}
  \label{eq:11}
  V \otimes W (\of{p}, \of{q}) := V(\of{p}) W(\of{q})
\end{equation}

If the basis vectors and one-forms are given as arguments to the
tensor the components of the tensor are returned.

It is also possible to reduce the dimensionality of a tensor by a
process called \emph{contraction} whereby an entire argument is
removed from the tensor by constructing an inner product of those
parts of the tensor with either a one form or a vector.

\section{Basis Transformations}
\label{sec:basis-transf}

Consider a point $P$ in a manifold $M$, where a vector basis
$\set{e_i}$ is defined, but we want to switch to a description in
another basis, $\set{e_{j'}}$, then in $T_P$ there is a linear
transformation, $\Lambda$ between the old and the new basis,
\begin{equation}
  \label{eq:12}
  e_{j'} = \Lambda^i_{j'} e_i 
\end{equation}
where $\Lambda$ is non-singular, but arbitrary. It is not a tensor,
since its indices refer to different bases. The transformation matrix has an inverse, $\Lambda^{k'}_j$ such that
\[ \Lambda^{k'}_j \Lambda^j_i = \delta^{k'}_{i'} \] We require the
inverse transformation for basis one-forms, as is evident from the
relation
\[ \of{\omega}^i e_{j'} = \delta^{i}_{k} \Lambda^k_{j'} =
\Lambda^i_{j'} \]

The general expression for the change of coordinates of a tensor takes
the form
\begin{equation}
  \label{eq:71}
  \tensor{A'}{^{u_1\dots u_l}_{r_1 \dots r_m}} = \Lambda_{t_1}^{u_1} \cdots \Lambda_{t_l}^{u_l} \Lambda_{r_1}^{q_1} \cdots \Lambda_{r_m}^{q_m} \tensor{A}{^{t_1 \dots t_l}_{q_1 \dots q_m}}
\end{equation}

\section{The Metric Tensor}
\label{sec:metric-tensor}

The metric tensor is a linear function defining the inner product of two vectors, such that
\begin{equation}
  \label{eq:13}
  g(V,U) = g(U,V) := U \vdot V
\end{equation}
where $g$ is the symmetric non-singular metric tensor, with components
in the basis $\set{e_i}$ being
\begin{equation}
  \label{eq:14}
  g_{ij} = e_i \vdot e_j
\end{equation}
The Euclidean metric is defined such that
\[ g_{ij} = \delta_{ij} \]
We can always form a metric, given transformations of the Cartesian basis,
\[ g_{i' j'} = \Lambda^k_{i'} g_{kl} \Lambda^l_{j'} \]
This can be reduced to an equation of the form
\[ g' = D O^{-1} g OD \] thanks to the existance of a theorem allowing
us to express any matrix as a product of an orthogonal matrix, $O$ and
a diagonal one, $D$. 

From this we can see that orthogonal matrices compose the set of
transformations between arbitrary Cartesian bases; they form a
continuous group, $O(n)$, which is the Euclidean symmetry group.

The Minkowski metric, with $g = \diag(-1, 1, 1, 1)$, also has
transformation matrices which form the Lorentz group, $L(n)$.

Some additional properties of the metric tensor are:
\begin{equation}
  \label{eq:15}
  \tensor{g}{^{ij}} \tensor{g}{_{jk}} = \delta^i_k
\end{equation}
and since $g^{ki} V_i = g^{ki} g_{ij} V^j = \delta^k_j V^j = V^k$, 
\begin{subequations}
  \begin{equation}
    \label{eq:16}
    V_i = g_{ij} V^j
  \end{equation}
  \begin{equation}
    \label{eq:17}
    v^j = g^{jk} V_k
  \end{equation}
\end{subequations}
allowing the metric tensor to act as an index raising and lowering
operator.

Defining a $\tensororder{0}{2}$ tensor on a manifold $M$ provides a
rich structure, and allow definitions of quantities such as distance
and curvature.
%%% Local Variables: 
%%% mode: latex
%%% TeX-master: "../project"
%%% End: 
