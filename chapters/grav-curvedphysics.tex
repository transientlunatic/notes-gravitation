Recall the definition of the Ricci tensor and the Ricci scalar,
\begin{equation}
  \tag{\ref{eq:69}}
  \tensor{R}{_{\beta \nu}} = \tensor{g}{^{\alpha \mu}} \tensor{R}{_{\alpha \beta \mu \nu}} = \tensor{R}{^{\mu}_{\beta \mu \nu}}
\end{equation}
and 
\begin{equation}
  \tag{\ref{eq:70}}
  R = \tensor{g}{^{\beta \nu}} \tensor{R}{_{\beta \nu}}
\end{equation}
Note that the Ricci tensor is symmetric; by differentiating equation
\eqref{eq:67},
\begin{equation}
  \label{eq:92}
  2 \tensor{R}{_{\alpha \beta \mu \nu, \lambda}} = \tensor{g}{_{\alpha \beta, \beta \mu \lambda}} - \tensor{g}{_{\alpha \mu, \beta \nu \lambda}} 
                                           +\tensor{g}{_{\beta \mu, \alpha \nu \lambda}} - \tensor{g}{_{\beta \nu, \alpha \mu \lambda}}
\end{equation}
Recalling that partial derivatives do commute,
\begin{equation}
  \label{eq:93}
  \tensor{R}{_{\alpha \beta \mu \nu, \lambda}} + \tensor{R}{_{\alpha \beta \lambda \mu, \nu}} + \tensor{R}{_{\alpha \beta \lambda, \mu}} = 0
\end{equation}
Since the Ricci tensor is derived in normal coordinates, where
$\Gamma^{\mu}_{\alpha \beta} = 0$, so partial and covariant
differentiation are equivalent, and so we get the Bianchi identities,
\begin{equation}
  \label{eq:93}
  \tensor{R}{_{\alpha \beta \mu \nu; \lambda}} + \tensor{R}{_{\alpha \beta \lambda \mu; \nu}} + \tensor{R}{_{\alpha \beta \lambda; \mu}} = 0
\end{equation}
Performing the Ricci contraction from equation \eqref{eq:69} on the
Bianchi identities,
\begin{equation}
  \label{eq:94}
  \tensor{R}{\beta \nu; \lambda} - \tensor{R}{_{\beta \lambda; \nu}} + \tensor{R}{^{\mu}_{\beta \nu \lambda; \mu}} = 0
\end{equation}
contracting these in turn,
\begin{equation}
  \label{eq:95}
  \tensor{G}{^{\alpha \beta}_{; \beta}} = 0
\end{equation}
which is the contracted Bianchi identity, with $\ten{G}$, the Einstein
tensor, defined
\begin{fequation}
  \tensor{G}{^{\alpha \beta}} = \tensor{R}{^{\alpha \beta}} - \half \tensor{g}{^{\alpha \beta}} R
\end{fequation}

\section{The equivalence principle}
\label{sec:equiv-princ}

One statement of the equivalence principle is
\begin{quotation}
  All local, freely falling, non-rotating laboratories are fully
  equivalent for the performance of all physical experiments.
\end{quotation}
This doesn't rule out the possibility that the complicated curvature
of spacetime somehow reduces in a locally inertial frame. To preclude
this we can reword the equivalence principle.
\begin{quotation}
  Any physical law which can be expressed in tensor notation in
  special relativity has exactly the same form in a locally inertial
  frame of curved spacetime.
\end{quotation}
This means that partial differentiation is basically just a special
case of covariant differentiation for a locally inertial frame, that
is, in tensor notation,
\begin{equation}
  \label{eq:96}
  \tensor{T}{^{\mu \nu}_{;\nu}} = 0
\end{equation}
This also tells us how matter is affected by spacetime; in SR a
particle moves along the timelike coordinate of a Minkowski diagram
when at rest, and from the strong equivalence principle this must be
the case with general relativity too; this picks out the curves
generated by the timelike coordinate of a locally inertial frame, that
is
\begin{quotation}
  Space tells matter how to move: free-falling particles move on
  timelike geodesics of the local spacetime.
\end{quotation}

% \section{Geodesics and gravity}
% \label{sec:geodesics-gravity}

% A free-falling particle moves along a geodesic, and is not being
% accelerated; if something gets in its way this motion is impeded, and
% the particle is accelerated away from the geodesic.

\section{Einstein's Equations}
\label{sec:einsteins-equations}

Newton's theory of gravity can be expressed in terms of a
gravitational field, $\phi$, and the force on a particle of mass $m$
is $f_i = -m \phi_{,i}$, and the source of the field has a mass
density $\rho$. The field equation connecting the two is
\begin{subequations}
\begin{equation}
  \label{eq:90}
  \tensor{\phi}{^{,i}_{,i}} = 4 \pi G \rho
\end{equation}
which is Poisson's equation; in a vacuum with a mass density $\rho$,
\begin{equation}
  \label{eq:91}
  \tensor{\phi}{^{,i}_{,i}} = 0
\end{equation}
\end{subequations}
The acceleration of free-falling particles can be described, from
equation \eqref{eq:75} as 
\begin{equation*}
  \dv[2]{\xi^i}{t} = - \tensor{\phi}{^{,i}_{,i}} \xi^j
\end{equation*}
comparing this to equation \eqref{eq:63} we see both are equations of
geodesic deviation, implying that $\tensor{R}{^{\mu}_{\alpha \nu
    \beta}} U^{\alpha} U^{\beta}$ is analogous to
$\tensor{\phi}{^{,i}_{,j}}$, where the indices of the Riemann tensor
are swapped using its symmetries. The velocities of the particles,
$U^{\alpha}$ and $U^{\beta}$ are arbitrary, so the $\ten{\phi}$ of
Poisson's equation is analogous to $\tensor{R}{_{\alpha \beta}} =
\tensor{R}{^{\mu}_{\alpha \mu \beta}}$, meaning a good guess at the
relativistic analogue of equation \eqref{eq:91} is
\begin{equation}
  \label{eq:97}
  \tensor{R}{_{\mu \nu}} = 0
\end{equation}
These are Einstein's vacuum equations for general relativity; if
$\tensor{R}{_{\mu \nu}} = 0$ then $R = \tensor{g}{^{\mu \nu}} R_{\mu
  \nu} = 0$, thus
\begin{equation}
  \label{eq:98}
  G_{\mu \nu} = R_{\mu \nu} - \half R g_{\mu \nu} = 0
\end{equation}
But what if there's matter? $\rho$ is frame-dependent, so the
energy-momentum tensor is more likely to be what the field is bound
to. While $R^{\mu \nu}= - \kappa T^{\mu \nu}$ seems plausible, but
equation \eqref{eq:96} implies that $\tensor{R}{^{\mu \nu}_{;
    \nu}}=0$, which, via the Bianchi identity implies $\tensor{R}{_{;
    \nu}} = 0$, and $(g_{\alpha \beta} T^{\alpha \beta})_{; \nu} = 0$,
so this would imply that all matter has constant density, which is not
the case. What about
\begin{equation}
  \label{eq:99}
  \tensor{G}{^{\mu \nu}} = - \kappa \tensor{T}{^{\mu \nu}}
\end{equation}
numerous experiments have shown this to describe physical
reality. These are the \textbf{Einstein field equations}; ten
second-order non-linear differential equations, which reduce to six independent equations when the Bianchi identities are used.

One variation of the field equations which is now being taken very
seriously is
\begin{equation}
  \label{eq:100}
  \tensor{G}{^{\mu \nu}} + \Lambda \tensor{g}{^{\mu \nu}} 
  = - \kappa \tensor{T}{^{\mu \nu}}
\end{equation}
which we can do because $\tensor{g}{_{\alpha \beta; \mu}}=0$. This
contains an extra term, $\Lambda$, the cosmological constant.

\section{The Newtonian limit}
\label{sec:newtonian-limit}

In the weak-field approximation we can take the spacetime around a
small object to be nearly Minkowskian, with
\begin{equation}
  \label{eq:101}
  \tensor{g}{_{\alpha \beta}} = \tensor{\eta}{_{\alpha \beta}} + \tensor{h}{_{\alpha \beta}}
\end{equation}
Thus $g_{\alpha \beta}$ is the result of a perturbation on flat
spacetime, and $\ten{h}$, which encodes the perturbation is a tensor
in Minkowskian space, and using this form in Einstein's equations
gives a mathematically tractable problem. In the Newtonian limit we
have $\abs{\phi} \ll 1$, and speeds $\abs{\vec{v}} \ll 1$, so this
implies $\abs{T^{00}} \gg \abs{T^{0i}} \gg \abs{T^{ij}}$. We then
identify $T^{00} = \rho + \mathcal{O}(\rho v^2)$. Matching the
resulting form of Einstein's equation with Newton's equation we fix
the constant $\kappa$, so, in geometrical units,
\begin{equation}
  \label{eq:102}
  G^{\mu \nu} = 8 \pi T^{\mu \nu}
\end{equation}
The solution in this approximation is then
\begin{equation}
  \label{eq:103}
  h^{00} = h^{11} = h^{22} = h^{33} = -2 \phi
\end{equation}
So the Newtonian metric is then
\begin{equation}
  \label{eq:104}
  \ten{g} \to \diag( -(1+ 2\phi), 1- 2\phi, 1-2\phi, 1 - 2 \phi)
\end{equation}
and its interval is then
\begin{equation}
  \label{eq:105}
  \dd{s^2} = -(1+2 \phi) \dd{t}^2 + (1- 2 \phi) (\dd{x}^2 + \dd{y}^2 + \dd{z}^2 )
\end{equation}
The geodesic equation is $\nabla_U U = 0$; this geodesic has an affine
parameter $\tau$, but rescaling, $\tau \to \tau/m$, we can express it
in terms of momentum, $p = m U$, so
\begin{equation}
  \label{eq:106}
  \nabla_p p = 0
\end{equation}
In component form this looks like
\begin{equation}
  \label{eq:107}
  p^{\alpha} \tensor{p}{^{\mu}_{,\alpha}} + \Gamma^{\mu}_{\alpha \beta} p^{\alpha} p^{\beta} = 0
\end{equation}
Restricting the motion to non-relativistic particles, $\abs{p^0} \gg \abs{p^i}$, and so
\begin{equation}
  \label{eq:108}
  m \dv{\tau} p^{\mu} + \Gamma_{00}^{\mu} (p^0)^2 = 0
\end{equation}
The 0-0 symbols in this metric and approximation are
\begin{subequations}
  \begin{align}
    \label{eq:110}
    \Gamma^0_{00} &= \phi_{,0} + \mathcal{O}(\phi^2) \\
\label{eq:111}
\Gamma_{00}^i &= -\half \tensor{(-2 \phi)}{_{,j}} \delta^{ij}
  \end{align}
\end{subequations}
Thus,
\begin{subequations}
  \begin{align}
    \label{eq:113}
    \dv{p^0}{\tau} &= -m \pdv{\phi}{\tau} \\
    \label{eq:114}
\dv{p^i}{\tau} &= -m \phi^{,i}
  \end{align}
\end{subequations}
Which is Newton's law of gravitation.

%%% Local Variables: 
%%% mode: latex
%%% TeX-master: "../project"
%%% End: 
