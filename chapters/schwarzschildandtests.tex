There are four major classical tests of general relativity
\begin{enumerate}
\item advance of pericentre
\item light deflection
\item redshift
\item time delay
\end{enumerate}

\section{Geodesics of the Schwarzschild metric}
\label{sec:geod-schw-metr}

The geodesics of a test particle in the Schwarzchild metric satisfy
the geodesic equation (\ref{eq:112}) with the proper time, $\tau$ as
the affine parameter,
\begin{equation}
  \label{eq:141}
  \dv{\tau}\qty(g_{\lambda\nu} \dv{x^{\nu}}{\tau} ) - \half \pdv{g_{\mu \nu}}{x^{\lambda}} \dv{x^{\mu}}{\tau} \dv{x^{\nu}}{\tau} = 0
\end{equation}
The metric is static, and symmetrical, so is independent of $t$ and
$\phi$, thus if $\lambda \in \set{0,3}$ the second term vanishes. The
metric is also orthogonal, so
\[ \dv{\tau} \qty( g_{tt} \dv{t}{\tau}) = \dv{\tau} \qty( g_{\phi \phi} \dv{\phi}{\tau}) = 0 \]
Integrating these gives
\[ g_{tt} \dv{t}{\tau} = \text{constant}, \qquad g_{\phi \phi} \dv{\phi}{\tau} = \text{constant} \]
The geodesic equation for $\lambda=2$ ($\theta$) follows from equation (\ref{eq:112})
\begin{equation}
  \label{eq:142}
  \dv{\tau} \qty( r^2 \dv{\theta}{\tau} ) - \half  \pdv{\theta} \qty( r^2 \sin[2](\theta) )\qty[ \dv{\phi}{\tau}]^2 = 0 
\end{equation}
which reduces to
\begin{equation}
  \label{eq:143}
  r^2 \dv[2]{\theta}{\tau} + 2 r \dv{r}{\tau} \dv{\theta}{\tau} - r^2 \sin(\theta) \cos(\theta) \qty(\dv{\phi}{\tau})^2 = 0
\end{equation}
This has a particular solution when $\theta=\pi/2$, which fixes the
orbit of a test particle to lie in the equatorial plane of the
coordinate system, thus \[ \dv{\theta}{\tau} = 0 \] We then also have
\begin{equation}
  \label{eq:144}
  \dv{t}{\tau} = \frac{k}{1 - \frac{2M}{r}}, \qquad \dv{\phi}{\tau} = \frac{h}{r^2}
\end{equation}
For $k$, $h$ constant. The geodesic differential equation for $r$ can
then be found from the general geodesic equation, 
\begin{equation}
  \label{eq:145}
  -1 = g_{tt} \qty( \dv{t}{\tau} )^2 + g_{rr} \qty(\dv{r}{\tau})^2 + g_{\phi \phi} \qty( \dv{\phi}{\tau} )^2
\end{equation}
which reduces to
\begin{equation}
  \label{eq:146}
  \qty( \dv{r}{\tau} )^2 = k^2 - 1 - \frac{h^2}{r^2} + \frac{2M}{r} \qty(1+ \frac{h^2}{r^2})
\end{equation}

\section{Planetary orbits in Newtonian gravity}
\label{sec:plan-orbits-newt}

In Newtonian mechanics a test mass orbiting another mass $M$ on the
equatorial plane has equations of motion of the form
\begin{equation}
  \label{eq:147}
  r^2 \dv{\phi}{t} = h , \qquad \dv[2]{r}{t} = - \frac{M}{r^2} + r \qty( \dv{\phi}{t} )^2
\end{equation}
To solve the second equation we normally change variables, $t \to u =
1/r$ and $t \to \phi$. We have \[ \dv{\phi}{t} = hu^2 \] so
\begin{equation}
  \label{eq:148}
  \dv{r}{t} = \dv{t} \qty(\frac{1}{u}) = - \frac{1}{u^2} \dv{u}{\phi} \dv{\phi}{t} = - h \dv{u}{\phi} 
\end{equation}
so
\begin{equation}
  \label{eq:149}
  \dv[2]{u}{\phi} = -u + \frac{M}{h^2}
\end{equation}
which has a solution
\begin{equation}
  \label{eq:150}
  u = \frac{M}{h^2} \qty( 1+ e \cos(\phi))
\end{equation}
which is an ellipse with eccentricity $e$, semi-major axis $a$, and a
focus at $r=0$. $h$ is related to the semi-latus rectum
\[ \ell = \frac{h^2}{M} = a(1-e^2) \]


\section{Advance of pericentre}
\label{sec:advance-pericentre}

Performing a similar procedure on equation (\ref{eq:146}), with $r \to
u$ and $\tau \to \phi$, we find
\begin{equation}
  \label{eq:151}
  h^2 \qty( \dv{u}{\phi})^2 = \qty(k^2 - 1) - h^2u^2 + 2 Mu \qty( 1+h^2u^2) 
\end{equation}
Differentiating this, and cancelling the common $\dv{u}{\phi}$,
\begin{equation}
  \label{eq:152}
  \dv[2]{u}{\phi}  = -u + \frac{M}{h^2} + 3 Mu^2
\end{equation}
This tiny $3Mu^2$ component is the additional contribution from GR;
for the Earth's orbit the ratio of this to the $M/h^2$ quantity is
around $3\e{-8}$. We can gain a good approximation to equation
(\ref{eq:152}) by replacing the $u$ by a $u^2$ term on the right hand
of the Newtonian equivalent, so
\begin{equation}
  \label{eq:153}
  \dv[2]{u}{\phi} = -u + \frac{M}{h^2} + 3 \frac{M^3}{h^4} \qty( 1 + 2e \cos(\phi) + e^2 \cos[2](\phi) )
\end{equation}
\begin{figure}[t] \centering
\begin{tikzpicture} 
  \begin{polaraxis}
    \addplot+[mark=none,domain=0:5000,samples=1200, muted-green, line width=1]
    {1/(1+0.6*cos(0.9*x))}; % equivalent to (x,{sin(..)cos(..)}), i.e. % the expression is the RADIUS
  \end{polaraxis}
\end{tikzpicture}
\caption{The precession of the pericentre.}
\end{figure}
decomposing $u$ into $u = u~N + u~{GR}$, and then subtracting the Newtonian component
\begin{equation}
  \label{eq:155}
  \dv[2]{u~{GR}}{\phi} = -u~{GR} + 3 \frac{M^3}{h^4} \qty(1+ 2e \cos(\phi) + e^2 \cos[2](\phi) )
\end{equation}
Noting
\begin{equation}
  \label{eq:154}
  \cos[2](\phi) = \half (1 + \cos(2 \phi) )
\end{equation}
then
\begin{equation}
  \label{eq:156}
  \dv[2]{u~{GR}}{\phi} + u~{GR} = 3 \frac{M^3}{h^4} \qty( 1 + \frac{e^2}{2} + 2e \cos(\phi) + \frac{e^2}{2} \cos(2 \phi) )
\end{equation}
The right-hand side has the form $A + B \cos(\phi) + C \cos(2 \phi)$ for constants $A,B,C$.
\begin{subequations}
  \begin{align}
    u~{GR} &= A \\ &= \half B \phi \sin(\phi) \\ &= - \frac{1}{3} C \cos(2 \phi)
  \end{align}
\end{subequations}
with the correction to the Newtonian orbit given by the sum of these
integrals. The first and third terms add a negligible constant, and a
tiny wiggle, but the second term is secular, so increases constantly.
Thus
\begin{equation}
  \label{eq:157}
  u = \frac{M}{h^2} \qty( 1+ e \cos(\phi) + \frac{3M^2}{h^2} e \phi \sin(\phi) )
\end{equation}
given that the $3M^2/h^2$ term is very small, and making small angle
approximations, $\cos(\beta)\approx 1$ and $\sin(\beta)\approx \beta$,
and recalling that
\[ \cos(\alpha-\beta) = \cos(\alpha) \cos(\beta) + \sin(\alpha) \sin(\beta) \]
we can rewrite equation (\ref{eq:157}) as
\begin{equation}
  \label{eq:158}
  u  =\frac{M}{h^2} \qty[1 + e \cos( \qty( 1 - \frac{3M^2}{h^2}) \phi )]
\end{equation}
Comparing this to the Newtonian analogue it's clear that the solution
is elliptical and periodic with a period
\begin{equation}
  \label{eq:159}
  P = \frac{2 \pi}{1 - 3M^2/h^2} > 2 \pi
\end{equation}
so the ellipse precesses, and the pericentre line advances by an
amount $\Delta$ in every orbit,
\begin{equation}
  \label{eq:160}
  \Delta = 2 \pi \qty( 1 - \frac{3M^2}{h^2})^{-1} - 2 \pi \approx \frac{6 \pi M^2}{h^2} = \frac{6 \pi M}{a(1-e^2)}
\end{equation}



\section{Gravitational light deflection}
\label{sec:grav-light-defl}

The deflection of light occurs in the Newtonian framework, provided
photons have mass, but is formally predicted by General relativity for
all particles with energy.

\subsection{Newtonian deflection}
\label{sec:newtonian-deflection}

Consider the path of a photon passing by a mass $M$; the orbit of the
photon will be a hyperbola with $M$ at one focus, so
\[ r(\phi) = \frac{ r~{min} (e+1) }{1+e \cos(\phi) } \] for $e>1$ the
eccentricity, and $r~{min}$ the closest distance the photon approaches
to $M$.
\begin{figure}[t] \centering
\begin{tikzpicture} 
  \begin{polaraxis}
    \addplot+[mark=none,domain=-100:100,samples=300, muted-green, line
    width=1]
    {0.2*(1.6+1)/(1+1.6*cos(x))}; % equivalent to (x,{sin(..)cos(..)}), i.e. % the expression is the RADIUS
    \draw [thick] (axis cs:90, 0.52) -- (axis cs:90, 0.7);
\draw [thick, bend right] (axis cs:90,0.7) to (axis cs:98.5,0.67);
\draw (axis cs:90,0.6) node[right] {$\frac{\Delta\pi}{2}$};
  \end{polaraxis}
\end{tikzpicture}
\caption{The hyperbolic trajectory of a photon.}
\end{figure}
Before and after the encounter the photon has asymptotic directions
\begin{equation}
  \label{eq:161}
  \phi = \pm \qty( \frac{\pi}{2} + \frac{\Delta \phi}{2})
\end{equation}
for $\Delta \phi$ the total deflection angle. Just like for an
elliptical orbit we require \[ r^2 \dv{\phi}{t} = h \] for $h$ constant,
\[ h^2 = M \ell = Ma (e^2 - 1) = Ma (e-1)(e+1) = M r~{min}(e+1) \]
with $\ell$ the semi-latus rectum.
The photon must also satisfy an energy equation,
\begin{equation}
  \label{eq:162}
  \half v^2 - \frac{M}{r} = \half \qty[ \qty( \dv{r}{t} )^2 + r^2 \qty( \dv{\phi}{t} )^2 ] - \frac{M}{r} = E~{tot}
\end{equation}
for the terms in the brackets the radial and transverse velocity
components. Since 
\[ \dv{r}{t} = \dv{r}{\phi} \dv{\phi}{t} \]
\begin{equation}
  \label{eq:163}
  \frac{h^2}{2r^2} \qty[ \frac{1}{r^2} \qty(\dv{r}{\phi})^2 + 1] - \frac{M}{r} = E~{tot}
\end{equation}
Then from the equation of the orbit,
\begin{equation}
  \label{eq:164}
  \dv{r}{\phi} = \frac{r~{min}(e+1)e \sin(\phi)}{(1+e \cos(\phi))^2} = \frac{r^2 e \sin(\phi)}{r~{min} (e+1)}
\end{equation}
Substituting for $h$ in the energy equation,
\begin{equation}
  \label{eq:165}
  E~{tot} = \frac{M(e-1)}{2 r~{min}}
\end{equation}
We can see that $r \to \infty$ when $\cos(\phi) = -1/e$, and letting
$v=c=1$, for $r \to \infty$ we have
\begin{equation}
  \label{eq:166}
  E~{tot} = \half
\end{equation}
Rearranging equation (\ref{eq:165}), and since $e\gg 1$
\begin{equation}
  \label{eq:167}
  e = 1 + \frac{r~{min}}{M} \approx \frac{r~{min}}{M}
\end{equation}
So the outgoing photon has an asymptotic direction
\begin{equation}
  \label{eq:168}
  \cos(\phi) = \cos( \frac{\pi}{2} + \frac{\Delta \phi}{2} ) = - \sin( \frac{\Delta \phi}{2}) = - \frac{M}{r~{min}}
\end{equation}
Since $\Delta \phi \ll 1$,
\begin{equation}
  \label{eq:169}
  \Delta \phi = \frac{2M}{r~{min}}
\end{equation}

\subsection{Relativistic deflection}
\label{sec:relat-defl}

The geodesics for a photon can be found in a similar way to those for
masses in section \ref{sec:geod-schw-metr}, but the affine parameter
is $s$, since a photon has a proper time $\tau = 0$. We find
\begin{subequations}
  \begin{align}
    \dv{t}{s} &= \frac{k}{1 - 2 \frac{M}{r}} \\ \dv{\phi}{s} &= \frac{h}{r^2}
  \end{align}
\end{subequations}
For the $\theta$ equation there is a particular solution
$\theta=\frac{\pi}{2}$, and also \[ \frac{\theta}{s}=0 \]
so
\begin{equation}
  \label{eq:170}
  \qty( \dv{r}{s})^2 = k^2 - \frac{h^2}{r^2} + \frac{2 M h^2}{r^3} 
\end{equation}
Making the change of variables $r \to u = 1/r$, and $s \to \phi$, then
\begin{equation}
  \label{eq:171}
  \dv[2]{u}{\phi} + u = 3 M u^2
\end{equation}
Ignoring the right hand side, a particular integral is 
\begin{equation}
  \label{eq:172}
  u = \frac{\cos(\phi)}{r~{min}}
\end{equation}
Following the same approach as before, and using equation
(\ref{eq:172}) to substitute into the right hand side of equation
(\ref{eq:171}),
\begin{equation}
  \label{eq:173}
  \dv[2]{u}{\phi} + u = \frac{3M}{r^2~{min}} \cos[2](\phi) = \frac{3M}{2 r^2~{min}} (1 + \cos(2 \phi))
\end{equation}
A particular integral for this approximation is then
\begin{equation}
  \label{eq:174}
  u = \frac{3M}{2 r^2~{min}} \qty(1 - \frac{1}{3} \cos(2 \phi) )
\end{equation}
and it follows that the general solution to equation (\ref{eq:171}) is
\begin{equation}
  \label{eq:175}
  u = \frac{\cos(\phi)}{r~{min}} + \frac{3M}{2 r^2~{min}} \qty( 1 - \frac{1}{3} \cos(2 \phi) )
\end{equation}
we can rewrite this using the asymptotic directions of the photon, so the outgoing photon has
\begin{equation*}
  \label{eq:176}
  u = \frac{\cos( \frac{\pi}{2} + \frac{\Delta \phi}{2})}{r~{min}} + \frac{3 M}{2 r^2~{min}} \qty[1-\frac{1}{3} \cos(\pi + \Delta \phi) ]
\end{equation*}
which simplifies to
\begin{equation}
  \label{eq:177}
    u = -\frac{\sin(\frac{\Delta \phi}{2})}{r~{min}} + \frac{3 M}{2 r^2~{min}} \qty[1-\frac{1}{3} \cos(\Delta \phi) ]
\end{equation}
and since $\Delta \phi \ll 1$,
\begin{equation}
  \label{eq:178}
  u = - \frac{\Delta \phi}{2 r~{min}} + \frac{2M}{r^2~{min}}
\end{equation}
Then setting $u=0$ (i.e. $r \to \infty$), we get the general relativisitic result,
\begin{equation}
  \label{eq:179}
  \Delta \phi = \frac{4 M}{r~{min}} \equiv \frac{4 GM}{c^2 r~{min}} = \frac{2 R~S}{r~{min}}
\end{equation}
This is twice the Newtonian deflection. If we take $r~{min}$ to be a
solar radius, and $M$ to be a solar mass, we find the deflection
caused by the sun to be $1.77$ arcseconds, which was observed by
Arthur Eddington during a solar eclipse.

\subsection{Lensing}
\label{sec:lensing}

\begin{figure}[b]
  \centering
  \begin{tikzpicture}[scale=0.8]
    \node (O) at (0,0) {$O$}; 
    \node (M) at (5,0) {$M$};
\node (P) at (5,2) {$P$};
\node (S) at (10,0) {$S$};
\draw (O) to (P); 
\draw [<->] (O) to (M);
\draw (2.5, 0) node [fill=white]  {$D_L$};
\draw [<->] (M) to (S);
\draw (7.5,0) node [fill=white] {$D_S-D_L$};
\draw [<->] (P) to (M);
\draw (5,1) node [fill=white] {$R~E$};
\node (S') at (10,4) {$S'$};
\draw [dotted] (P) to (S');
\draw [bend right] (0:1) to (22:1);
\node at (11:1.5) {$\theta~E$};
\draw [help lines] (P) to (S);
\draw [bend left] (S) -- +(0:-1) to +(-22:-1);
\node at ($(S)+(-11:-1.5)$) {$\beta$};
\draw [bend left]($(P)+(22:1)$) to ($(P)+(-22:1)$);
\draw ($(P)+(0:.75)$) node {$\alpha$};
  \end{tikzpicture}
  \caption{The geometrical arrangement leading to gravitational lensing about a mass $M$.}
  \label{fig:lensing}
\end{figure}

Consider a light ray from a distant source (see figure
\ref{fig:lensing}) being deflected through an angle $\alpha$ by a mass
$M$. The angle of deflection is small, so can be approximated by a
thin-lensing situation, with the lens lying along the line $OP$, with
$P$ a distance $R~E$ from $M$. The ray path is then approximated by
$OP$ and $PS$. By symmetry all points on the locus described by $R~E$
produce lensing, so a ring is formed (figure \ref{fig:lensing-2}),
which is called an \emph{Einstein ring}.

The ring has an angular radius $\theta~E$, which can be found by
considering the distance $D_L$ to the lens and $D_S$ to the source,
from the observer at $O$. Then (see figure \ref{fig:lensing}) 
\[ \theta~E + \beta = \alpha \]
and 
\[ \theta~E = \frac{R~E}{D_L} \]
also
\[ \beta = \frac{R~E}{D_S - D_L} = \theta~E \frac{D_L}{D_S-D_L} \]
Then, making appropriate substitutions,
\begin{equation*}
  \label{eq:180}
  \theta~E + \theta~E \frac{D_L}{D_S-D_L} = \theta~E \frac{D_S}{D_S - D_L} = \frac{4M}{R~E} = \frac{4M}{D_L \theta~E}
\end{equation*}
Thus
\begin{equation}
  \label{eq:181}
  \theta~E = \sqrt{ \frac{4M (D_S-D_L)}{D_SD_L} }
\end{equation}
then, substituting $x = D_L/D_S$,
\begin{equation}
  \label{eq:182}
  \theta~E = \sqrt{ \frac{4M(1-x)}{D_S x}}
\end{equation}

If the lensing mass is not a spherically symmetric distribution
exactly aligned between the observer and the source the ring is not
produced, but instead an (odd) number of images are formed with an
angular separation on the order of $\theta~E$.

We can make some simplifications to get some convenient equations for
lensing. Suppose the source is a distant quasar lensed by a foreground
galaxy, then
\begin{equation}
  \label{eq:183}
  \theta~E \approx 3" \sqrt{ \frac{M}{10^{12} M_{\odot}} \frac{10^9\, \text{pc}}{D_S} \frac{(1-x)}{x} }
\end{equation}
This regime is \emph{strong lensing}, and separations are on the order
of a few arcseconds. Several hundreds of examples of strong lenses exist.

If the source objects are within the Galaxy,
\begin{equation}
  \label{eq:184}
  \theta~E \approx 0.9\, \text{mas} \sqrt{ \frac{M}{M_{\odot}} \frac{10\,\text{kpc}}{D_S} \frac{(1-x)}{x}}
\end{equation}
So the images are close enough that resolving them is difficult (but
within the power of e.g. GAIA), and this regime is \emph{gravitational
  microlensing}.

\begin{figure}[b]
  \centering
  \begin{tikzpicture}[scale=0.6]
%\clip (0, -1) rectangle (10, 2.5);
    \node (O) at (0,0) {$O$}; 
    \coordinate (M) at (5,0);
\coordinate (P) at (5,2);
\coordinate (P') at (5,-2) {};
\node (S) at (10,0) {$S$};
\draw (O) to (P); \draw (O) to (P');
\coordinate (S') at (10,4);
\draw [dotted] (P) to (S');
\draw [bend right] (0:1) to (22:1);
\node at (11:1.5) {$\theta~E$};
\draw [help lines] (P) to (S) (P') to (S); 
\draw [bend left]($(P)+(22:1)$) to ($(P)+(-22:1)$);
\draw ($(P)+(0:.75)$) node {$\alpha$};
\draw [ultra thick] (S) ellipse (2 and 4);
\fill (M) circle (0.05) node [above] {$M$};

  \end{tikzpicture}
  \caption{The production of an \emph{Einstein ring}.}
  \label{fig:lensing-2}
\end{figure}

\section{Gravitational redshift}
\label{sec:grav-redsh-1}

Suppose light is emitted with the coordinates $(t_e,r_e)$ in the
Schwarzschild metric, and travels along a radial null geodesic to
reach an observer at an event $(t_o, r_o)$, then
\begin{equation}
  \label{eq:185}
  \dd{s}^2 = 0 = - \qty( 1 - \frac{2M}{r} ) \dd{t}^2 + \frac{\dd{r}^2}{1 - 2M/r}
\end{equation}
thus
\begin{equation}
  \label{eq:186}
  \int_{t_e}^{t_o} \dd{t} = t_o - t_e = \int_{r_e}^{r_o} \frac{\dd{r}}{1 - 2M/r}
\end{equation}
A wave with rest-frame frequency $\nu_0$ will have wavecrests leave
$r_e$ at $t_e$ and $t_e + \Delta t_e$, which reach $r_o$ at $t_o$ and
$t_o+\Delta t_e$, so the proper time difference between emissions is
\begin{equation}
  \label{eq:187}
  \Delta \tau_e = \Delta t_e \sqrt{1 - \frac{2M}{r_e}} \equiv \frac{1}{\nu_e} \equiv {\lambda_e}
\end{equation}
and between detections is
\begin{equation}
  \label{eq:188}
  \Delta \tau_o = \Delta t_e \sqrt{1 - \frac{2M}{r_o}} \equiv \frac{1}{\nu_o} \equiv \lambda_o
\end{equation}
so the redshift, $z$, is
\begin{equation}
  \label{eq:189}
  z = \frac{\lambda_o - \lambda_e}{\lambda_e} = \sqrt{ \frac{1-2M/r_o}{1-2M/r_e}} - 1 = \sqrt{\frac{r_e(r_o-R~S)}{r_o(r_e-R~S)}}-1
\end{equation}
for $R~S$ the Schwarzschild radius of the mass $M$.

\section{Gravitational time delay}
\label{sec:grav-time-delay}

Gravitational time delay has two constributions; a geometric one, and
a gravitational one, the Shapiro effect.

Taking the Schwarzschild invariant interval, and introducing 
\begin{equation}
  \label{eq:190}
  r = R \qty(1+\frac{M}{2R})^2
\end{equation}
then assuming a weak gravitational field, with $M \ll r$ and so $M \ll R$,
\begin{equation}
  \label{eq:191}
  r approx R \qty(1+ \frac{M}{R})
\end{equation}
and noting \[ 1 - \frac{2M}{r} = \frac{r-2M}{r} \]
we can produce the Schwarzschild metric in the form
\begin{equation}
  \label{eq:192}
\begin{split}
  \dd{s}^2 = - \qty( \frac{1-M/R}{1+M/R} ) \dd{t}^2 + \qty(\frac{1+M/R}{1-M/R}) \dd{r}^2 \\
+R^2 \qty( 1 + \frac{M}{R})^2 \qty[ \dd{\theta}^2 + \sin[2](\theta) \dd{\phi}^2]
\end{split}
\end{equation}
Using the binomial expansion,
\[ (1+x)^n \approx 1+ nx \]
and, noting that to first order $\dd{r} = \dd{R}$,
\begin{equation}
  \label{eq:194}
  \begin{split}
    \dd{s}^2 = - \qty(1 - \frac{2M}{R}) \dd{t}^2 \\
+ \qty(1+\frac{2M}{R}) \qty[ \dd{R}^2 + R^2 \dd{\theta}^2 + R^2 \sin[2](\theta) \dd{\phi}^2]
  \end{split}
\end{equation}
Then, defining the Cartesian coordinates,
\begin{equation}
  \label{eq:195}
  X = R \sin(\theta) \cos(\theta), \quad Y=R \sin(\theta) \sin(\phi), \quad Z = R \cos(\theta)
\end{equation}
and introducing the gravitational potential,
\[ \psi = - \frac{GM}{R} \]
The interval becomes
\begin{equation}
  \label{eq:196}
  \dd{s}^2 = -(1+2\psi) \dd{t}^2 + (1-2\psi) [\dd{X}^2 + \dd{Y}^2 + \dd{Z}^2]
\end{equation}

Consider a photon which propagates from a source at $A$ to an observer
at $B$ past a deflecting mass $M$, which is small, so we assume the
trajectory is straight in space, and align the coordinates so the
$z$-axis is along the line of propagation, so $\dd{x}=\dd{y}=0$, so
\begin{equation}
  \label{eq:197}
  \dd{s}^2 = - (1+2 \phi) \dd{t}^2 + (1-2 \psi) \dd{z}^2
\end{equation}
Since $\dd{s}^2 = 0$, to first order,
\begin{equation}
  \label{eq:198}
  \dd{t}^2 = \frac{1-2\psi}{1+2\psi} \dd{z}^2 \approx (1-2\psi)^2 \dd{z}^2
\end{equation}
Then $\int \dd{t} = \int(1-2\phi) \dd{z}$, so
\begin{equation}
  \label{eq:199}
  t~B - t~A = (z~B - z~A) - 2 \int_{z~A}^{z~B} \psi(z) \dd{z}
\end{equation}
With the second term being the time delay due to gravity,
\begin{equation}
  \label{eq:200}
  \delta t~{grav} = -2 \int_{z~A}^{z~B} \psi(z) \dd{z} \equiv -\frac{2}{c^3} \int_{z~A}^{z~B} \psi(z) \dd{z}
\end{equation}
%%% Local Variables: 
%%% mode: latex
%%% TeX-master: "../project"
%%% End: 
