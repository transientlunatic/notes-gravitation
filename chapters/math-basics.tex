\section{The topology of $\mathbb{R}^n$}
\label{sec:topology-mathbbrn}

The space $\rn$ is the usual $n$-dimensional space of vector algebra,
where a position can be described by a real $n$-tuple, $(x_1, x_2,
\dots, x_n)$. The local topology of $\rn$ is centred around the
concept of a neighbourhood of a point in $\rn$, and this allows the definition of a distance function between points, $\vec{x}$ and $\vec{y}$ in $\rn$,
\begin{equation}
  \label{eq:1}
  d(\vec{x},\vec{y}) = \qty[ (x_1 - y_1)^2 + (x_2 - y_2)^2 + \cdots + (x_n - y_n)^2 ]^{\half}
\end{equation}

\begin{definition}[Neighbourhood]
  A neighbourhood of a point $x$ in $X$ is a set $N(x)$ containing an
  open set which contains $x$.
\end{definition}
A family of neighbourhoods induces a notion of ``nearness'' to $x$.

A neighbourhood of radius $r$ of the point $\vec{x}$ in $\rn$ is thus
the set of points $N_r(\vec{x})$ with a distance less than $r$ from
$\vec{x}$. Such a neighbourhood is \emph{discrete} if each point has a
neighbourhood containing no other points; which is clearly untrue for
$\rn$, which is thus continuous.

A set of points $S$ in $\rn$ is \emph{open} if every point $x \in S$ has a
neighbourhood contained within $S$. The interval $a<x<b$ is an open
neighbourhood, but $a \leq x < b$ is closed, as the points $a=x$ have
neighbourhoods partially outside the set.

\begin{definition}[Hausdorff Space]
  A space is Hausdorff if any two distinct points in the space posess
  disjoint neighbourhoods.
\end{definition}

We can draw a line between two points in $\rn$, as we can have
neighbourhoods about each point which do not intersect, which is the
Hausdorff property of the space.

Using the distance function $d(\vec{x}, \vec{y})$ to define these
properties induces a topology on a space, enabling the definition of
open sets in the space, which have the properties
\begin{enumerate}
\item if $O_1$ and $O_2$ are open sets, then so is their intersection, $O_1 \cap O_2$
\item the union of any collection of open sets is open.
\end{enumerate}
To make (1) apply to all sets the empty set must be defined as open, and for (2) to work $\rn$ as a whole must be open.

\section{Mappings}
\label{sec:mappings}

A map from a space $M$ to a space $N$ is a rule which associates
elements $x \in M$ to elements $y \in N$. The simplest map is a
function $f:\mathbb{R} \to \mathbb{R}$ taking $x \mapsto f(x)$. 

Some terminology for mappings of the form $f:M \to N$,
\begin{description}
\item[image] the set, $T$ of elements from the set $N$ which are mapped to from elements of $S \subset M$. Denoted $f(S)$.
\item[inverse-image] the set $S$
\end{description}
There are a number of types of mapping.
\begin{description}
\item[Many-to-one] a map where multiple elements of $S$ map to the same element of $T$ 
\item[one-to-one] a map where only one element of $S$ maps to each element of $T$
\item[inverse] a map which takes $T \to S$, and is only well-defined for one-to-one maps
\end{description}

Let $f:M \to N$ and $g: N \to P$, then the operation of function composition is defined as 
\[ g \circ f : M \to P \]

A map $f:M \to N$ is continuous at $x \in M$ if any open set of $N$
containing $f(x)$ contains the image of an open set of $M$, provided
$M$,$N$ are topological spaces. $f$ is continuous on $M$ if it is continuous at all the points in $M$.

If $f(x_1, \dots, x_n)$ is a function which is defined on an open
region $S$ of $\rn$ the it is \emph{differentiable of class} $C^k$ if
all its partial derivatives of the order less or equal to $k$ exist
and are themselves continuous on $S$; $f$ is a $C^k$ function.

\section{Real Analysis}
\label{sec:real-analysis}

A real function is analytic if, at $x=x_0$ it has a Taylor expansion
about $x_0$ which converges to $f(x)$ in a neighbourhood of $x_0$,
{\small
\begin{align}
  \label{eq:2}
  f(x) &= f(x_0) + (x-x_0) \eval{\dv{f}{x}}_{x_0} 
        + \half (x-x_0)^2 \eval{\dv[2]{f}{x}}_{x_0}
       %&\quad + \frac{1}{3!} (x-x_0)^3 \eval{\dv[3]{f}{x}}_{x_0} 
       + \cdots
\end{align}
} This clearly requires that $f$ be infinitely differentiable, but
this is not a guarantee that $f$ is analytic, e.g. $\exp(-1/x^2)$
which is non-analytic as a real function, but becomes analytic with a
singularity at $z=0$ as a complex function.

A real-valued function $g(\cdots)$ defined on an open region $S$ of $\rn$ is square-integrable if 
\begin{equation}
  \label{eq:3}
  \int_S \qty[ g(x_1, \dots, x_n)]^2 \dd{x_1 \dots \dd{x_n}}
\end{equation}
exists. A square-integrable function can be approximated by an analytic function $g'$ such that the integral of $(g - g')^2$ over $S$ can be made arbitrarily small.

A $C^{\infty}$ function need not be analytic, so an analytic function
is denoted $C^{\omega}$.

An operator $A$ acts on functions defined on $\rn$, and takes one function, $f$, into another one, $A(f)$. Examples are multiplication, differentiation, and fixed-kernel integration.
The commutator of two operators is defined
\begin{equation}
  \label{eq:4}
  [A,B](f) = (AB-BA)(f)
\end{equation}

%%% Local Variables: 
%%% mode: latex
%%% TeX-master: "../project"
%%% End: 
