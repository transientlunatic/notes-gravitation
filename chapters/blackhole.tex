
\section{Beyond white dwarfs and neutron stars}
\label{sec:beyond-white-dwarfs}

Is it always justified to assume that the centre of a star has a
finite pressure and density? Degenerate remnants represent a class of
objects where the central pressure is only maintained by quantum
degeneracies; what happens if the gravitational collapse is capable of
overbalancing these? This becomes a problem if the coordinate radius
of the star satisfies $r=2M$, at the Schwarzschild surface. At this
point the Schwarzschild metric misbehaves.

\section{The Schwarzschild surface}
\label{sec:schw-surf}

In order to understand the misbehaviour of the metric when $g_{rr} =
2M$, and whether this is a physical problem with spacetime, or a
coordinate problem, we can consider a particle falling across the
$r=2M$ surface. Suppose the particle starts at a coordinate radius
$R>2M$, and is released at a coordinate time $t=0$, and proper time
$\tau=0$ in the particle's rest frame. Then
\[ \qty( \dv{r}{\tau} )^2 = k^2 - 1 - \frac{h^2}{r^2} + \frac{2M}{r} \qty( 1+ \frac{h^2}{r^2}) \]
and
\[ \dv{\phi}{\tau} = \frac{h}{r^2} \]
For a radial trajectory $h=0$, and 
\[ \qty( \dv{r}{\tau})^2 = k^2 -1 + \frac{2M}{r} \]
and as it is initially at rest,
\[ k^2-1=- \frac{2M}{R} \implies \qty( \dv{r}{\tau})^2 = \frac{2M}{r}-\frac{2M}{R}   \]
Given that the particle is falling inwards,
\[ \dd{\tau} = - \frac{\dd{r}}{\sqrt{\frac{2M}{r} - \frac{2M}{R}}} \]
So the elapsed proper time experienced by the particle as it falls
from $r=R$ to $r=2M$ is
\begin{equation}
  \label{eq:231}
  \Delta \tau = \int_{2M}^R \frac{\dd{r}}{\sqrt{\frac{2M}{r} - \frac{2M}{R}}}
\end{equation}
This integral is finite, and so the particle reaches the surface in a
finite time. What about the interval in coordinate time?
\[ \dv{t}{\tau} = \frac{k}{1-\frac{2M}{r}} = \frac{\sqrt{1- \frac{2M}{R}}}{1-\frac{2M}{r}} \]
so
\[ \dd{t} = - \frac{\sqrt{1 - \frac{2M}{R}}}{\sqrt{\frac{2M}{r} - \frac{2M}{R}} \qty( 1 - \frac{2M}{r})} \dd{r} \]
and so
\begin{equation}
  \label{eq:233}
  \Delta t = \int_{2M}^R  \frac{\sqrt{1 - \frac{2M}{R}}}{\sqrt{\frac{2M}{r} - \frac{2M}{R}} \qty( 1 - \frac{2M}{r})} \dd{r}
\end{equation}
This integral diverges as $r \to 2M$, so it takes an infinite amount
of coordinate time to reach the radius. The same is true for a
photon. Thus an external observer sees the particle taking an infinite
time to fall into the black hole.

Clearly, the problem with the metric at the Schwarzschild surface is
merely a coordinate singularity. There are no physical consequences in
the local frame of the particle as it crosses the singularity, but
this singularity is the event horizon, and it has effects on the
future movement of the particle. Once this surface is crossed it is
impossible to return, as all trajectories cause the particle to move
to a smaller coordinate radius, unless the particle can exceed the
speed of light.

\section{Inside the event horizon}
\label{sec:inside-event-horizon}

\begin{figure}[t]
  \centering
   \begin{tikzpicture}[scale=0.5]
\tikzset{help lines/.style={draw = gray!20}}
% \draw [pattern=north west lines, pattern color=gray!50]  (0,3.75) to  (7.5,3.75) to (3.75, 0) -- cycle ;
% \draw [pattern=north east lines, pattern color=gray!50]  (0,-3.75) to  (7.5,-3.75) to (3.75, 0) -- cycle ;
%\draw [ultra thick] (0,-3.75) -- (7.5,-3.75); 
\draw [ultra thick] (0,1.75) -- (1.65,1.75) node [midway, above] {$r=0$}; 
%\draw (3.75,-4.1) node {$r=0$}; \draw (3.75,4.1) node {$r=0$};
\draw (2.5, 1.9) node {$i^+$}; 
\draw (-1, 0) node {$r=0$};
\draw (0.2, -4.1) node {$i^-$}; \draw (3.8, 0) node {$i^0$};
\draw (3, 1) node {$\mathcal{I}^+$}; \draw (3, -1) node {$\mathcal{I}^-$};
%\draw (0,1.75) -- (0,-2.5);
\clip (0,-4.1) rectangle (4,1.75);
\begin{scope}
   \begin{axis}[mark=none, xmin=-0, xmax=90, 
                           ymin=-90, ymax=90, 
                           anchor=origin, height=10cm, width=5cm,
                           xlabel=$U$, ylabel=$V$,
                           axis lines=none]
     \foreach \u in {-0.5, -0.375, ..., 0.5}{
       \addplot[mark=none, help lines, domain=-79:79]{atan(\u - tan(x))  +  atan(\u + tan(x)) };
       }
   \end{axis}
   \begin{axis}[mark=none, xmin=-90, xmax=90, 
                           ymin=-0, ymax=90, 
                           anchor=origin, height=5cm, width=10cm,
                           xlabel=$U$, ylabel=$V$,
                           axis lines=none,
                           rotate around={-90:(axis cs:0,0)},
                           ]
     \foreach \u in {-0.5, -0.375, ...,0.5}{
       \addplot[mark=none, help lines, domain=-79:79]{atan(\u - tan(x))  +  atan(\u + tan(x)) };
       }
       \clip [thick] (axis cs: -80,0)--(axis cs: 0, 80)--(axis cs: 80,0)--(axis cs:0,-80)--cycle;
       \draw [ultra thick] (axis cs: -80,0)--(axis cs: 0, 80)--(axis cs: 80,0)--(axis cs:0,-80)--cycle;

       \addplot [fill=black, ultra thick, mark=none, domain=-79:79] {atan(0.15 - tan(x))  +  atan(0.15 + tan(x)) } -- cycle;

   \end{axis}
 \end{scope}
\draw [ultra thick] (0,1.75) -- (1.65,1.75); 
%\draw (0,1.75) -- (0,-4);



\end{tikzpicture}
%%% Local Variables: 
%%% mode: latex
%%% TeX-master: "../project"
%%% End: 

%%% Local Variables: 
%%% mode: latex
%%% TeX-master: "../project"
%%% End: 

  \caption{The Penrose diagram for a real black hole formed from the collapse of a massive star.}
  \label{fig:real-black-hole}
\end{figure}

\begin{figure}[b]
  \centering
   \begin{tikzpicture}[scale=0.5]
\tikzset{help lines/.style={draw = gray!20}}
\begin{scope}
   \begin{axis}[mark=none, xmin=-90, xmax=90, 
                           ymin=-90, ymax=90, 
                           anchor=origin, height=10cm, width=10cm,
                           xlabel=$U$, ylabel=$V$,
                           axis lines=none]
     \foreach \u in {-0.5, -0.375, ..., 0.5}{
       \addplot[mark=none, help lines, domain=-79:79]{atan(\u - tan(x))  +  atan(\u + tan(x)) };
       }
   \end{axis}
   \begin{axis}[mark=none, xmin=-90, xmax=90, 
                           ymin=-90, ymax=90, 
                           anchor=origin, height=10cm, width=10cm,
                           xlabel=$U$, ylabel=$V$,
                           axis lines=none,
                           rotate around={-90:(axis cs:0,0)},
                           ]
     \foreach \u in {-0.5, -0.375, ...,0.5}{
       \addplot[mark=none, help lines, domain=-79:79]{atan(\u - tan(x))  +  atan(\u + tan(x)) };
       }
       \clip [thick] (axis cs: -80,0)--(axis cs: 0, 80)--(axis cs: 80,0)--(axis cs:0,-80)--cycle;
       \draw [ultra thick] (axis cs: -80,0)--(axis cs: 0, 80)--(axis cs: 80,0)--(axis cs:0,-80)--cycle;
   \end{axis}
 \end{scope}
\begin{scope}[xshift=7.5cm, anchor=origin]
   \begin{axis}[mark=none, xmin=-90, xmax=90, 
                           ymin=-90, ymax=90, 
                           anchor=origin, height=10cm, width=10cm,
                           xlabel=$U$, ylabel=$V$,
                           axis lines=none]
     \foreach \u in {-0.5, -0.375, ..., 0.5}{
       \addplot[mark=none, help lines, domain=-79:79]{atan(\u - tan(x))  +  atan(\u + tan(x)) };
       }

   \end{axis}
   \begin{axis}[mark=none, xmin=-90, xmax=90,
                           ymin=-90, ymax=90, 
                           anchor=origin, height=10cm, width=10cm,
                           xlabel=$U$, ylabel=$V$,
                           axis lines=none,
                           rotate around={-90:(axis cs:0,0)},
                           ]
     \foreach \u in {-0.5, -0.375, ...,0.5}{
       \addplot[mark=none, help lines, domain=-79:79]{atan(\u - tan(x))  +  atan(\u + tan(x)) };
       }
       \draw [ultra thick] (axis cs: -80,0)--(axis cs: 0, 80)--(axis cs: 80,0)--(axis cs:0,-80)--cycle;
   \end{axis}
 \end{scope}
 \draw [pattern=north west lines, pattern color=gray!50]  (0,3.75) to  (7.5,3.75) to (3.75, 0) -- cycle ;
 \draw [pattern=north east lines, pattern color=gray!50]  (0,-3.75) to  (7.5,-3.75) to (3.75, 0) -- cycle ;
\draw [ultra thick] (0,-3.75) -- (7.5,-3.75); \draw [ultra thick] (0,3.75) -- (7.5,3.75); 
\draw (3.75,-4.1) node {$r=0$}; \draw (3.75,4.1) node {$r=0$};
\draw (7.9, 4.1) node {$i^+$}; \draw (7.9, -4.1) node {$i^-$}; \draw (11.8, 0) node {$i^0$};
\draw (10.5, 2) node {$\mathcal{I}^+$}; \draw (10.5, -2) node {$\mathcal{I}^-$};
\end{tikzpicture}
%%% Local Variables: 
%%% mode: latex
%%% TeX-master: "../project"
%%% End: 

  \caption{The Penrose conformal diagram for a Schwarzchild black hole.}
  \label{fig:schwarz-conformal}
\end{figure}

Recall that the interval, $\dd{s}^2$ between neighbouring events---
$(t,x,y,z)$ and $(t+\dd{t}, x+\dd{x}, y+\dd{y}, z+\dd{z})$ in some
coordinate system---can be null $\dd{s}^2=0$, spacelike ($\dd{s}^2>0$),
or timelike ($\dd{s}^2<0$). In a spacelike interval there will be a
Lorentz frame, $S'$ in which the events occur at the same coordinate
time, so $\dd{t}=0$. If $\dd{s}^2>0$ the two events cannot lie on the
worldline of a material particle, as an observer in $S'$ would see it
in two places at the same time, violating causality.

Consider neighbouring events for a particle at rest inside the event
horizon, with $\dd{r} = \dd{\theta} = \dd[\phi] = 0$, but $\dd{t} \neq
0$, then
\[ \dd{s}^2 = - \qty( 1 - \frac{2M}{r} ) \dd{t}^2 \] Since $r<2M$ it
follows that this interval must be positive, so cannot lie on the
worldline of the particle, and so a particle cannot remain stationary;
in effect $t$ and $r$ have changed role as coordinate labels.

To overcome the misbehaviour of the coordinates at a radius $r=2M$ we
introduce a new time coordinate, 
\begin{equation}
  \label{eq:234}
  \tilde{t} = t+2M \log( \frac{r}{2M} - 1 )
\end{equation}
taking $\dd{\theta} = \dd{\phi} = 0$,
\begin{equation}
  \label{eq:235}
  \dd{s}^2 = - \qty(1 - \frac{2M}{r}) \dd{\tilde{t}}^2 + \frac{4M}{r} \dd{r} \dd{\tilde{t}} + \qty( 1 + \frac{2M}{r}) \dd{r}^2
\end{equation}
which has no coordinate singularity at $r=2M$ (but has one at $r=0$,
which is a \emph{physical} singularity).

We can then obtain the equations of the null cones by setting
$\dd{s}^2=0$, dividing through by $\dd{r}^2$, and solving for
$\dv{\tilde{t}}{r}$, leaving a quadratic equation with roots at
\begin{equation}
  \label{eq:236}
  \dv{\tilde{t}}{r} = \qty{-1, \frac{1 + 2M/r}{1-2M/r}}
\end{equation}
As $r$ approaches the Schwarzschild radius the light cones start to
tip over, and at $r=R~s$ the null cone has a vertical edge, so all
timelike geodesics then point inwards.

\section{Luminosity and Hawking radiation}
\label{sec:luminosity}

\subsection{Redshifting}
\label{sec:redshifting}

Just how ``black'' is a black hole? Shouldn't we see the light emitted
by the star just before it collapsed eternally frozen at the event
horizon? Light from the collapsing star is redshifted as it climbs out
of the star's gravity field, with the redshift, $z$, 
\begin{equation}
  \label{eq:237}
  z \equiv \frac{\lambda~o - \lambda~e}{\lambda~e} = \sqrt{\frac{1 - 2M/r~o}{1-2M/r~e}} -1  = \sqrt{\frac{r~e (r~o-R~s)}{r~o (r~e - R~s)}} - 1
\end{equation}
which diverges as $r~e \to R~s$. The bolometric luminosity of the
star, compared to its constant luminosity $L~c$, ignoring relativistic
effects is
\[ L(t_0) = \frac{L~c}{(1+z)^2} \] This can be understood by
considering the energy of each photon received by the observer is
redshifted by a factor $(1+z)$, and the arrival times are increased by
the same factor, so the luminosity is reduced by the square. The light
ray, on a null geodesic, satisfies
\[ \int_{t~e}^{t~o} \dd{t} = \int_{r~e}^{r~o} \frac{\dd{r}}{1-2M/r} \equiv  \int_{r~e}^{r~o} \frac{\dd{r}}{1-R~s/r} \]
Then
\[ t~o - t~e = r~o - r~e - R~s \log( \frac{r~o - R~s}{r~e - R~s} ) \]
and, taking $t_0 = 0$
\[ \log( \frac{r~e - R~s}{r~o - R~s} ) = \qty[\frac{t~o - (r~o - r~e)}{R~s}]\]
\[ \implies \frac{r~e - R~s}{r~o - R~s} =
\frac{r~o}{r~e}\frac{1}{(1+z)^2} \propto \exp( - \frac{t~o}{R~s} ) \]
The, reintroducing the speed of light, $c$,
\[ \frac{L(t~o)}{L~c} \propto \exp( - \frac{c t~o}{R~s} ) \] The
luminosity of the star falls off exponentially in the time taken for
light to cross the Schwarzschild radius.

\subsection{Hawking radiation}
\label{sec:hawking-radiation}

The effects of quantum field theory in a curved spacetime do, however,
allow black holes to have a very faint luminosity---a process known as
Hawking radiation. This basically relies on the Uncertainty Principle
allowing energy to be ``borrowed'' from the vacuum on a timescale
$\Delta t$, where
\[ \Delta t = \frac{\hbar}{\Delta E} \] This allows the temporary
production of virtual photon pairs; if this occurs on the edge of the
event horizon the negative energy member of the pair may fall into the
black hole, leaving the positive energy photon free to propagate
outside the black hole, and the negative energy one free to propagate
inside the black hole. This radiation has a black body profile at
infinity,
\[ E_{\infty} = \frac{h}{8 \pi M} \] This means that a black hole has
a thermodynamic temperature, and since it is emitting energy, it must
be losing mass. The luminosity of a black hole can be derived using
the Stefan-Boltzmann law, and is proportional to the product of the
area of the event horizon, $A$, and the fourth power of the
temperature, $T^4$, so
\begin{align}
  A &= 4 \pi R^2 = 4 \pi (2M)^2 = 16 \pi M^2 \nonumber\\
\implies L & \propto M^{-2} \nonumber\\
\implies \dot{M} & \propto M^{-2} \nonumber
\end{align}
Thus the lifetime of the black hole is $\tau \propto M^3$, and eventually we find
\begin{equation}
  \label{eq:240}
  \qty( \frac{\tau}{10^{10}\,\text{yr}} ) = \qty( \frac{M}{10^{12}\,\kilogram} )^3
\end{equation}

\subsection{Black hole thermodynamics}
\label{sec:black-hole-therm}

The Hawking area theorem, which predates the work on Hawking
radiation, states that
\[ \dv{A}{t} \ge 0 \] which has an analogy in thermodynamics:
entropy. The discovery of Hawking radiation firmed-up this analogy, by
attaching a temperature to a black hole. For a Schwarzschild black
hole,
\[ \dd{A} = 32 \pi M \dd{M} \]
or
\[ \dd{M} = \frac{1}{32 \pi M} \dd{A} = \frac{\hbar}{8 \pi k M} \dd{\qty(\frac{Ak}{4 \hbar})} \]
Thus
\[ \dd{E} = T \dd{S} \] This only applies to a classical black hole,
however, since Hawking radiation reduces the mass and hence the area
of the event horizon.


\section{Rotating black holes}
\label{sec:rotating-black-holes}

The treatment of rotating reference frames is substantially more
complicated than static ones, but an example of a more general black
hole than the Schwarzschild one, the Kerr black hole, requires this
treatment, as it is rotating.

The Kerr metric is characterised by the constants $M$ and $J$, which
can be found by requiring that the metric must reproduce Newtonian
behaviour in the weak-field limit. $M$ is the Newtonian mass of the
star, and $J$ is the magnitude of the total angular momentum. Writing
$a \equiv J/M$, the form of the metric is
\begin{align}
  \dd{s}^2 &= - \frac{\Delta - a^2 \sin[2](\theta)}{\rho^2} - 4a \frac{Mr \sin[2](\theta)}{\rho^2} \dd{t} \dd{\phi} \nonumber\\
 & \qquad {}+ \frac{(r^2+a^2)^2 -a^2 \Delta \sin[2](\theta)}{\rho^2} \sin[2](\theta) \dd{\phi}^2 \nonumber\\
& \qquad {}+ \frac{\rho^2}{\Delta} \dd{r}^2 + \rho^2 \dd{\theta}^2
\end{align}
having defined $\Delta = r^2 - 2Mr + a^2$ and $\rho^2 = r^2 + a^2
\cos[2](\theta)$. This metric is clearly non-diagonal, and this
introduces the phenomenon of frame-dragging.

\subsection{Conservation of four momentum on geodesics}
\label{sec:cons-four-moment}

A material particle has a geodesic equation 
\[ \dv{v^{\alpha}}{\tau} + \Gamma^{\alpha}_{\beta \delta} v^{\beta} v^{\delta} = \qty( \pdv{v^{\alpha}}{x^{\beta}} v^{\beta} + \Gamma^{\alpha}_{\beta \delta} v^{\beta} v^{\delta} ) = v^{\beta} \tensor{v}{^{\alpha}_{;\beta}} = 0 \]
The mixed form is
\[ v^{\alpha} v_{\beta;\alpha} = 0 \] Introducing the four momentum,
$p^{\alpha} = m v^{\alpha}$, for $m$ the particle rest mass, then
\begin{equation}
  \label{eq:238}
  p^{\alpha} p_{\beta;\alpha} = 0
\end{equation}
Or,
\[ p^{\alpha}p_{\beta,\alpha} = \Gamma^{\gamma}_{\beta \alpha} p^{\alpha} p_{\gamma} = \half g^{\gamma \nu} \qty( g_{\nu \beta,\alpha} + g_{\nu \alpha, \beta} - g_{\alpha \beta, \nu}) p^{\alpha} p_{\gamma} \]
Which, after contraction and index permutation reduces to
\[ p^{\alpha} p_{\beta,\alpha} = \half g_{\nu \alpha, \beta} p^{\nu} p^{\alpha} \]
which can then be expressed as
\begin{equation}
  \label{eq:239}
  m v^{\alpha} \pdv{p_{\beta}}{x^{\alpha}} = m \dv{x^{\alpha}}{\tau} \pdv{p_{\beta}}{x^{\alpha}} = m \dv{p_{\beta}}{\tau} = \half g_{\nu \alpha,\beta} p^{\nu} p^{\alpha}
\end{equation}
If all the components of the metric are independent of the coordinate
$x^{\beta}$ hen the right hand side is zero, implying that
$p^{\alpha}$ is constant along the geodesic.

\subsection{Frame dragging}
\label{sec:frame-dragging}

The components of the Kerr metric are independent of $\phi$, but a
material particle moving on a geodesic conserves the $p_{\phi}$
component of its four momentum. The contravariant component $p^{\phi}$
is
\[ p^{\phi} = g^{\phi \alpha} p_{\alpha} = g^{\phi \phi} p_{\phi} +
g^{\phi t} p_t \] In an orthogonal metric the second term would be
zero, but the Kerr metric is non-orthogonal. Similarly, for $p^t$,
\[ p^t = g^{t\alpha} p_{\alpha} =  g^{tt} p_t + g^{t \phi} p_{\phi} \]
A particle with $p_{\phi}=0$ will have
\[p^t = m \dv{r}{\tau}, \qquad p^{\phi} = m \dv{\phi}{\tau} \]
So
\[ \dv{\phi}{t} = \frac{p^{\phi}}{p^t} = \frac{g^{\phi t}}{g^{tt}}
\neq 0 \] Thus a distance observer sees an angular velocity: an
irrotational particle free-falling in a Kerr metric gains angular
momentum.

%%% Local Variables: 
%%% mode: latex
%%% TeX-master: "../project"
%%% End: 
