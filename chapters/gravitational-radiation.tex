
\section{Non-stationarity}
\label{sec:non-stationarity}

A non-stationary metric has a metric with some dependence on the time
coordinate, $t$, and an important consequence of these metrics is that
they permit gravitational radiation.

\section{Weak fields}
\label{sec:weak-fields}

In the absence of energy and matter spacetime is flat and
Mikowskian. We can introduce a perturbation, however, to make the
metric ``nearly-flat'', with the form
\begin{equation}
  \label{eq:229}
  g_{\alpha \beta} = \eta_{\alpha \beta} + h_{\alpha \beta}
\end{equation}
Where $\eta_{\alpha \beta} = \diag(-1,1,1,1)$ is the Minkowski metric,
and $\abs{h_{\alpha \beta}} \ll 1$. This spacetime still obeys Lorentz
transformations, $\Lambda$, so
\begin{equation}
  \label{eq:1}
  g'_{\alpha \beta} = \Lambda_{\alpha'}^{\mu} \Lambda_{\beta'}^{\nu} g_{\mu \nu} =  \Lambda_{\alpha'}^{\mu} \Lambda_{\beta'}^{\nu} \eta_{\mu \nu} +  \Lambda_{\alpha'}^{\mu} \Lambda_{\beta'}^{\nu} h_{\mu \nu} = \eta_{\alpha \beta}' + h_{\alpha \beta}'
\end{equation} 
Gauge transformations can also be applied, with the form
\begin{equation}
  \label{eq:2}
  x'^{\alpha} = x^{\alpha} + \xi^{\alpha}(x^{\beta})
\end{equation}
for $\xi^{\alpha}$ functions of the coordinates $\set{x^{\alpha}}$, and so
\begin{equation}
  \label{eq:3}
  \Lambda_{\alpha}^{\beta} = \delta^{\alpha}_{\beta} + \xi_{,\beta}^{\alpha}
\end{equation}
We can demand that the $\xi^{\alpha}$ are small, so
\[ \abs{\xi^{\alpha}_{,\beta}} \ll 1 \quad \forall \alpha,\beta\]
Thus by the chain rule
\begin{equation}
  \label{eq:5}
  \Lambda_{\alpha}^{\gamma} = \delta^{\alpha}_{\gamma} - \Lambda^{\gamma}_{\beta} \Lambda^{\beta}_{\alpha} \approx \delta^{\alpha}_{\gamma} - \xi_{,\gamma}^{\alpha} 
\end{equation}

If the unprimed coordinate system is nearly Lorentz, then
\begin{align*}
  g'_{\alpha \beta} &= \Lambda_{\alpha}^{\nu} \Lambda_{\beta}^{\nu} g_{\mu \nu} \\
&= \qty( \delta_{\alpha}^{\mu} \delta_{\beta}^{\nu} - \xi_{,\alpha}^{\mu} \delta_{\beta}^{\nu} - \xi^{\nu}_{,\beta} \delta_{\alpha}^{\mu} ) \eta_{\mu \nu} + \delta_{\alpha}^{\mu} \delta_{\beta}^{\nu} h_{\mu \nu} \\
&= \nu_{\alpha \beta} + h_{\alpha \beta} -\xi_{\alpha, \beta} - \xi_{\beta , \alpha}
\end{align*}
This has the same form as equation (\ref{eq:229}) provided that
\[ h'_{\alpha \beta} = h_{\alpha \beta} - \xi_{\alpha, \beta} - \xi_{\beta, \alpha} \]

Clearly the new primed system is also nearly Lorentz.

\section{Einstein's equations in weak fields}
\label{sec:einst-equat-weak}

In the nearly Lorentz system we have a restricted number of coordinate
transforms which can be used so that the resulting system is still
nearly Lorentz.

To the first order for small perturbations the Riemann-Christoffel
tensor for a nearly-flat space is 
\begin{equation}
  \label{eq:7}
  \tensor{R}{_{\alpha \beta \gamma \delta}} = \half \qty( h_{\alpha \delta, \beta \gamma}+ h_{\beta \gamma, \alpha \delta} - h_{\alpha \gamma, \beta \delta} - h_{\beta \delta,\alpha \gamma} )
\end{equation}
The Ricci tensor can then be obtained,
\begin{equation}
  \label{eq:8}
  R_{\mu \nu} = \half \qty( h^{\alpha}_{\mu, \nu \alpha} + h^{\alpha}_{\nu, \mu \alpha} - h_{\mu \nu,\alpha}^{,\alpha} - h_{, \mu \nu} )
\end{equation} for $h \equiv h^{\alpha}_{\alpha} = \eta^{\alpha \beta} h_{\alpha \beta}$. Also the Ricci scalar,
\begin{equation}
  \label{eq:9}
  R = \eta^{\alpha \beta} R_{\alpha \beta}
\end{equation}
allowing the Einstein tensor to be found,
\begin{align}
  G_{\mu \nu} &= R_{\mu \nu} - \half \eta_{\mu \nu} R \nonumber\\
&= \half \qty[h_{\mu \alpha,\nu}^{,\alpha} + h_{\nu \alpha,\mu}^{, \alpha} - h_{\mu \nu,\alpha}^{,\alpha} - h_{, \mu \nu} - \eta_{\mu \nu} \qty( h_{\alpha \beta}^{,\alpha \beta} - h_{,\beta}^{,\beta})]
\end{align}
By rescaling the metric perturbations,
\begin{equation}
  \label{eq:10}
  \bar{h}_{\mu \nu} \equiv h_{\mu \nu} - \half \eta_{\mu \nu} h
\end{equation}
then 
\begin{equation}
  \label{eq:11}
  G_{\mu \nu} = - \half \qty[ \bar{h}_{\mu \nu, \alpha}^{,\alpha} + \eta_{\mu \nu}\bar{h}_{\alpha \beta}^{, \alpha \beta} - \bar{h}_{\mu \alpha, \nu}^{,\alpha} - \bar{h}_{\nu \alpha, \mu}^{, \alpha}]
\end{equation}
Since
\[ G_{\mu \nu} = 8 \pi T_{\mu \nu} \] it follows
\begin{equation}
  \label{eq:12}
  - \bar{h}_{\mu \nu,\alpha}^{,\alpha} - \eta_{\mu \nu}^{,\alpha \beta} + \bar{h}_{\mu \alpha, \nu}^{, \alpha} + \bar{h}_{\nu \alpha, \mu}^{,\alpha} = 16 \pi T_{\mu \nu}
\end{equation}
We can always find a gauge in which the last three terms are
zero---the Lorentz gauge, which is equivalent to adopting the
coordinate system in which \[ \bar{h}^{\mu \alpha}_{, \alpha} = 0 \]
(i.e. the metric with the divergence of perturbations equal to zero), thus
\begin{equation}
  \label{eq:13}
  - \bar{h}_{\mu \nu, \alpha}^{,\alpha} = 16 \pi T_{\mu \nu}
\end{equation}

\subsection{Solutions in free space}
\label{sec:solutions-free-space}

Solutions in free space will be solutions of 
\begin{equation}
  \label{eq:14}
   \eta^{\alpha \alpha } \bar{h}_{\mu \nu, \alpha \alpha} = \bar{h}_{\mu \nu, \alpha}^{,\alpha} = 0
\end{equation}
which written in slightly more familiar notation is
\begin{equation}
  \label{eq:15}
  \qty( - \pdv[2]{t} + \nabla^2 ) \bar{h}_{\mu \nu} = 0
\end{equation}
moving out of geometrised units, setting $\eta^{00} = -c^{-2}$, then
\begin{equation}
  \label{eq:16}
   \qty( - \pdv[2]{t} + c^2\nabla^2 ) \bar{h}_{\mu \nu} = 0
\end{equation}
This has the form of a wave equation.

\section{Plane wave solutions}
\label{sec:plane-wave-solutions}

The simplest solutions to a wave equation are plane waves, which in
this case will take the form
\begin{equation}
  \label{eq:17}
  \bar{h}_{\mu \nu} = \Re\qty(A_{\mu \nu} \exp( i k_\alpha x^{\alpha}) )
\end{equation}
for $A_{\mu \nu}$ the amplitude of the waves, and $k_{\alpha}$ the
wavevector. $A_{\mu \nu}$ is a symmetric tensor, by fortune of
$\bar{h}_{\mu \nu}$ being symmetric, and this reduces the 16 distinct
components to 10. Next we have
\begin{equation}
  \label{eq:18}
  \tensor{\bar{h}}{_{\mu \nu, \alpha}^{,\alpha}} = \eta^{\alpha \sigma} \bar{h}_{\mu \nu, \alpha \sigma} = 0
\end{equation}
and so \[ k^{\alpha} k_{\alpha} = 0 \] Thus the wavevector is
null. The frequency is
\[ \omega = k^t = \qty( k_x^2 + k_y^2 + k_z^2)^{\half} \]
We also have a Lorentz gauge condition,
\[ \tensor{\bar{h}}(^{\mu \alpha}_{, \alpha} = \qty(\bar{h}_{\mu}^{\alpha})_{, \alpha} = 0 \]
thus
\begin{equation}
  \label{eq:19}
  A_{\mu \alpha} k^{\alpha} = 0
\end{equation}
So the wave amplitudes are orthogonal to the wavevector. These are
four equations, so we reduce from ten to six amplitude components. We
can then introduce a gauge fix to remove further terms,
\[ h_{\mu \nu}^{\text{new}} = h_{\mu \nu}^{\text{old}} - \xi_{\mu,\nu} - \xi_{\nu,\mu} \]
Such that
\begin{equation}
  \label{eq:20}
  G_{\mu \nu} = h^{\text{new}, \alpha}_{\mu \nu,\alpha}
\end{equation}
This transformation requires vector components which satisfy
\begin{equation}
  \label{eq:21}
  \qty( - \pdv[2]{t} + \nabla^2 ) \xi^{\mu} = \tensor{\bar{h}}{^{\text{old} \mu \nu}_{, \nu}}
\end{equation}
This doesn't define $\xi^{\mu}$ uniquely, and we can still add new
vector quantities to this, provided the same relations hold. This has
given another four equations, and so the number of free parameters in
the amplitude is now just two.

If we restrict the form of $A_{\mu \nu}$ to satisfy
\begin{equation}
  \label{eq:22}
  A^{\mu}^{\mu} = \eta^{\mu \nu} A_{\mu \nu} = 0, \quad A_{\alpha \beta}u^{\beta} = 0
\end{equation}
then the gauge choice is that of the Transverse-Traceless gauge. In a
background Lorentz frame with $u^{\beta} = \delta_t^{\beta}$ this
implies $A_{\alpha t} = 0 \ \forall \alpha$.

If the spatial axes are oriented so that the wave travels in the
positive $z$-direction then
\[ A_{\alpha z} = 0 \ \forall \alpha \]
and so
\begin{equation}
  \label{eq:23}
  \trt{\bar{h}}_{\mu \nu} = \trt{A}_{\mu \nu} \cos[ \omega(t-z) ]
\end{equation}
Thus
\begin{equation}
  \label{eq:24}
  \trt{h}_{\mu \nu} = \trt{B}_{\mu \nu} \cos[ \omega(t-z) ]
\end{equation}
where $\ten{B}$ has constant components.

The implications of all of the reductions are that in the TT gauge the
amplitude takes the form
\begin{equation}
  \label{eq:25}
  \trt{A}_{\mu \nu} =
  \begin{bmatrix}
     0 & 0      & 0       & 0 \\
0      & A_{xx} & A_{xy}  & 0 \\
0      & A_{xy} & -A_{xx} & 0 \\
0      & 0      & 0       & 0
  \end{bmatrix}
\end{equation}

\section{Free particles}
\label{sec:free-particles}

Consider a background Lorentz frame where a particle is initially at
rest, and the TT gauge is chosen. The particle has a geodesic
trajectory given by
\[ \dv{u^{\beta}}{\tau} + \Gamma^{\beta}_{\mu \nu} u^{\mu} u^{\nu} = 0 \]
The initial acceleration of the particle is then
\[ \qty( \dv{u^{\beta}}{\tau})_0 = - \Gamma^{\beta}_{tt} = -\half \eta^{\alpha \beta} (h_{\alpha t,t + h_{t \alpha, t} - h_{t t, \alpha}} )\]
But in the TT gauge $A_{\alpha t}=0$, so $\bar{h}_{\alpha t} = 0$, and $A_{\mu}^{\mu} = 0$, so $\bar{h} = \bar{h}_{\mu}^{\mu} = 0$, so $h_{\alpha t} = 0 \ \forall \alpha$. Thus
\[ \qty( \dv{u^{\beta}}{\tau})_0 = 0 \] Thus a particle initially at
rest stays at rest. Instead consider the proper distance, between
particles at $x=0$ and $x=\epsilon$.

\begin{equation*}
  \label{eq:26}
  \Delta \ell = \int \abs{ g_{\alpha \beta} \dd{x^{\alpha}} \dd{x^{\beta}}}^{\half} = \int_0^{\epsilon} \abs{g_{xx}}^{\half} \approx \epsilon \sqrt{g_{xx}(x=0)} 
\end{equation*}
Thus
\begin{equation}
  \label{eq:27}
  \Delta \ell \approx \qty[ 1 + \half \trt{h}_{xx} (x=0)] \epsilon
\end{equation}
Thus a passing gravitational wave produces a change in the proper
distance between test particles.

Now consider the geodesic deviation. Let $\xi^{\alpha}$ be the vector
connecting the test particles, then for a weak field
\begin{equation}
  \label{eq:28}
  \pdv[2]{\xi^{\alpha}}{t} = R^{\alpha}_{\mu \nu \beta} u^{\mu} u^{\nu} \xi^{\beta}
\end{equation}
The particles are initially at rest, so $u^{\mu} = (1,0,0,0)$ and
$\xi^{\beta} = (0, \epsilon, 0, 0)$. Thus
\begin{equation}
  \label{eq:29}
  \pdv[2]{\xi^{\alpha}}{t} = \epsilon R^{\alpha}_{ttx} = - \epsilon R^{\alpha}_{txt}
\end{equation}
The RC tensor is then
\begin{subequations}
  \begin{align}
    R^x_{txt} &= \eta^{xx} R_{xtxt} = - \half \trt{h}_{xx,tt} \\
R^y_{txt} &= \eta^{yy} R_{ytxt} = - \half \trt{h}_{xy,tt}
  \end{align}
\end{subequations}
Thus
\begin{subequations}
\begin{align}
  \label{eq:30}
  \pdv[2]{t} \xi^x &= \half \epsilon \pdv[2]{t} {\trt{h}_{xx}} \\
\pdv[2]{t} \xi^y &= \half \epsilon \pdv[2]{t} \trt{h}_{xy}
\end{align}
\end{subequations}

If we assemble a ring of test particles we can measure the
polarisation of a gravitational wave. Say we now assemble the test
particles at $x = \epsilon \cos(\theta)$ and at $y = \epsilon
\sin(\theta)$, then
\begin{subequations}
  \begin{align}
    \pdv[2]{t} \xi^x &= \half \epsilon \cos(\theta) \pdv[2]{t} \trt{h}_{xx} + \half \epsilon \sin(\theta) \pdv[2]{t} \trt{h}_{xy} \\
\pdv[2]{t} \xi^y &=  \half \epsilon \cos(\theta) \pdv[2]{t} \trt{h}_{xy} - \half \epsilon \sin(\theta) \pdv[2]{t} \trt{h}_{xx}
  \end{align}
\end{subequations}
The solutions to these equations take the form
\begin{subequations}
  \begin{align}
    \xi^x &= \epsilon \cos(\theta) \qty( 1 + \half \trt{B}_{xx} \cos(\omega t)) + \half \epsilon \sin(\theta) \trt{B}_{xy} \cos(\omega t) \\
\xi^y &= \epsilon \sin(\theta) \qty( 1 - \half \trt{B}_{xx} \cos(\omega t) ) + \half \epsilon \cos(\theta) \trt{B}_{xy} \cos(\omega t)
  \end{align}
\end{subequations}
These describe the equations of ellipses. As a wave passes by the
circular ring of test particles it deforms it into an ellipse, and
then back.

\begin{figure}[h]
  \centering
  \begin{tikzpicture}
  \begin{axis}[ height=4cm, width=4cm, axis equal, samples=100, xticklabels={}, yticklabels={}]
    %\addplot [domain=0:360, variable=\t, thick, gray ] ({cos(t)}, {sin(t)});

    \foreach \a in {-180,-140,...,180}{
    \addplot [domain=0:360, variable=\t, thick, muted-blue] ({cos(t)*(1+0.4*cos(\a)) }, {sin(t)*(1-0.4*cos(\a))});
                                                     };
  \end{axis}
\end{tikzpicture}
\begin{tikzpicture}
\begin{axis}[ height=4cm, width=4cm, axis equal, samples=100, xticklabels={}, yticklabels={}]
    %\addplot [domain=0:360, variable=\t, thick, gray ] ({cos(t)}, {sin(t)});

    \foreach \a in {0,40,...,360}{
    \addplot [domain=0:360, variable=\t, thick, muted-green] 
    ({cos(t) + 0.5*sin(t)*0.4*cos(\a) }, {sin(t) +0.5*cos(t)*0.4*cos(\a) });
                                                     };
  \end{axis}
\end{tikzpicture}
%%% Local Variables: 
%%% mode: latex
%%% TeX-master: "../project"
%%% End: 

  \caption{The polarisations of gravitational radiation. On the left
    $B_{xy} = 0$, and on the right $B_{xx} = 0$, producing the $+$ and
    $\times$ polarisation states.}
  \label{fig:polarisation}
\end{figure}

Gravitational radiation has two distinct polarisation states which are
illustrated in figure \ref{fig:polarisation}, and from this it is
clear that gravitational radiation is invariant under a 180 degree
rotation around its direction of propagation, compared to the 360
degree rotation for electromagnetic radiation, or the 720 degree
rotation. This can be understood in terms of the gauge bosons of the
fields.

In general, the classical radiation field of a particle with spin $S$
is invariant under a rotation of $360^{\circ}/S$. A photon has spin
$S=1$, so clearly a graviton must be spin $S=2$.

\section{Gravitational wave amplitude}
\label{sec:grav-wave-ampl}

We always expect the received gravitational radiation at the earth to
have a very small amplitude, thanks to the inverse square
relationship, even if the metric where it was produced was a strong
field metric. In such a situation $\abs{h_{\alpha \beta}}\sim 1$,
close to the source, when $r \sim M$, but at any distance $r$,
\begin{equation}
  \label{eq:32}
  \abs{H_{\alpha \beta}} \sim \frac{M}{R}
\end{equation}
Even the formation of a blackhole in the Andromeda Galaxy would only produce perturbations around 
\[ h_{\alpha \beta} \sim 6\e{-19} \] (and even this is a severe
over-estimate). Hence, detecting gravitational radiation is a severe
technological challenge.


\section{Quadrupolarity}
\label{sec:quadrupolarity}

In electromagnetic theory the dominant form of radiation is produced
by electric dipole radiation, where the luminosity output has the form
\begin{equation}
  \label{eq:33}
  L~{ed} \propto e^2 \vec{a}^2 \propto e^2
\end{equation}
for an acceleration of $\vec{\alpha}$ and distance $\vec{d}$ between
two particles of charge $e$. This is proportional to the second time
derivative of the magnetic dipole moment, 
\[ L~{ed} \propto \ddot{\mu} \]

In gravitation the equivalent of the electric dipole is the mass
dipole, and the mass dipole moment is
\begin{equation}
  \label{eq:34}
  \vec{d} = \sum_{A_i} m_i \vec{x}_i
\end{equation}
for $m_i$ the rest mass, and $\vec{x}_i$ the position of the $i$th
particle. The total linear momentum, $\vec{p}$, is then
\begin{equation}
  \label{eq:35}
  \vec{p} = \vec{d} - \sum_i m_i \vec{x}_i
\end{equation}
Total linear momentum is conserved, so there cannot be mass dipole
radiation from any source. The magnetic dipole corresponds to the
total angular momentum, and this too is conserved, so the mass dipole
has zero luminosity. Quadrupole radiation is the lowest order form
possible.

%%% Local Variables: 
%%% mode: latex
%%% TeX-master: "../project"
%%% End: 