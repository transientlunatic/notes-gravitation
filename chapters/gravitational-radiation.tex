
\section{Non-stationarity}
\label{sec:non-stationarity}

A non-stationary metric has a metric with some dependence on the time
coordinate, $t$, and an important consequence of these metrics is that
they permit gravitational radiation.

\section{Weak fields}
\label{sec:weak-fields}

In the absence of energy and matter spacetime is flat and
Mikowskian. We can introduce a perturbation, however, to make the
metric ``nearly-flat'', with the form
\begin{equation}
  \label{eq:229}
  g_{\alpha \beta} = \eta_{\alpha \beta} + h_{\alpha \beta}
\end{equation}
Where $\eta_{\alpha \beta} = \diag(-1,1,1,1)$ is the Minkowski metric,
and $\abs{h_{\alpha \beta}} \ll 1$. This spacetime still obeys Lorentz
transformations, $\Lambda$, so
\begin{equation}
  \label{eq:1}
  g'_{\alpha \beta} = \Lambda_{\alpha'}^{\mu} \Lambda_{\beta'}^{\nu} g_{\mu \nu} =  \Lambda_{\alpha'}^{\mu} \Lambda_{\beta'}^{\nu} \eta_{\mu \nu} +  \Lambda_{\alpha'}^{\mu} \Lambda_{\beta'}^{\nu} h_{\mu \nu} = \eta_{\alpha \beta}' + h_{\alpha \beta}'
\end{equation} 
Gauge transformations can also be applied, with the form
\begin{equation}
  \label{eq:2}
  x'^{\alpha} = x^{\alpha} + \xi^{\alpha}(x^{\beta})
\end{equation}
for $\xi^{\alpha}$ functions of the coordinates $\set{x^{\alpha}}$, and so
\begin{equation}
  \label{eq:3}
  \Lambda_{\alpha}^{\beta} = \delta^{\alpha}_{\beta} + \xi_{,\beta}^{\alpha}
\end{equation}
We can demand that the $\xi^{\alpha}$ are small, so
\[ \abs{\xi^{\alpha}_{,\beta}} \ll 1 \quad \forall \alpha,\beta\]
Thus by the chain rule
\begin{equation}
  \label{eq:5}
  \Lambda_{\alpha}^{\gamma} = \delta^{\alpha}_{\gamma} - \Lambda^{\gamma}_{\beta} \Lambda^{\beta}_{\alpha} \approx \delta^{\alpha}_{\gamma} - \xi_{,\gamma}^{\alpha} 
\end{equation}

If the unprimed coordinate system is nearly Lorentz, then
\begin{align*}
  g'_{\alpha \beta} &= \Lambda_{\alpha}^{\nu} \Lambda_{\beta}^{\nu} g_{\mu \nu} \\
&= \qty( \delta_{\alpha}^{\mu} \delta_{\beta}^{\nu} - \xi_{,\alpha}^{\mu} \delta_{\beta}^{\nu} - \xi^{\nu}_{,\beta} \delta_{\alpha}^{\mu} ) \eta_{\mu \nu} + \delta_{\alpha}^{\mu} \delta_{\beta}^{\nu} h_{\mu \nu} \\
&= \nu_{\alpha \beta} + h_{\alpha \beta} -\xi_{\alpha, \beta} - \xi_{\beta , \alpha}
\end{align*}
This has the same form as equation (\ref{eq:229}) provided that
\[ h'_{\alpha \beta} = h_{\alpha \beta} - \xi_{\alpha, \beta} - \xi_{\beta, \alpha} \]

Clearly the new primed system is also nearly Lorentz.

\section{Einstein's equations in weak fields}
\label{sec:einst-equat-weak}

In the nearly Lorentz system we have a restricted number of coordinate
transforms which can be used so that the resulting system is still
nearly Lorentz.

To the first order for small perturbations the Riemann-Christoffel
tensor for a nearly-flat space is 
\begin{equation}
  \label{eq:7}
  \tensor{R}{_{\alpha \beta \gamma \delta}} = \half \qty( h_{\alpha \delta, \beta \gamma}+ h_{\beta \gamma, \alpha \delta} - h_{\alpha \gamma, \beta \delta} - h_{\beta \delta,\alpha \gamma} )
\end{equation}
The Ricci tensor can then be obtained,
\begin{equation}
  \label{eq:8}
  R_{\mu \nu} = \half \qty( h^{\alpha}_{\mu, \nu \alpha} + h^{\alpha}_{\nu, \mu \alpha} - h_{\mu \nu,\alpha}^{,\alpha} - h_{, \mu \nu} )
\end{equation} for $h \equiv h^{\alpha}_{\alpha} = \eta^{\alpha \beta} h_{\alpha \beta}$. Also the Ricci scalar,
\begin{equation}
  \label{eq:9}
  R = \eta^{\alpha \beta} R_{\alpha \beta}
\end{equation}
allowing the Einstein tensor to be found,
\begin{align}
  G_{\mu \nu} &= R_{\mu \nu} - \half \eta_{\mu \nu} R \nonumber\\
&= \half \qty[h_{\mu \alpha,\nu}^{,\alpha} + h_{\nu \alpha,\mu}^{, \alpha} - h_{\mu \nu,\alpha}^{,\alpha} - h_{, \mu \nu} - \eta_{\mu \nu} \qty( h_{\alpha \beta}^{,\alpha \beta} - h_{,\beta}^{,\beta})]
\end{align}
By rescaling the metric perturbations,
\begin{equation}
  \label{eq:10}
  \bar{h}_{\mu \nu} \equiv h_{\mu \nu} - \half \eta_{\mu \nu} h
\end{equation}
then 
\begin{equation}
  \label{eq:11}
  G_{\mu \nu} = - \half \qty[ \bar{h}_{\mu \nu, \alpha}^{,\alpha} + \eta_{\mu \nu}\bar{h}_{\alpha \beta}^{, \alpha \beta} - \bar{h}_{\mu \alpha, \nu}^{,\alpha} - \bar{h}_{\nu \alpha, \mu}^{, \alpha}]
\end{equation}
Since
\[ G_{\mu \nu} = 8 \pi T_{\mu \nu} \] it follows
\begin{equation}
  \label{eq:12}
  - \bar{h}_{\mu \nu,\alpha}^{,\alpha} - \eta_{\mu \nu}^{,\alpha \beta} + \bar{h}_{\mu \alpha, \nu}^{, \alpha} + \bar{h}_{\nu \alpha, \mu}^{,\alpha} = 16 \pi T_{\mu \nu}
\end{equation}
We can always find a gauge in which the last three terms are
zero---the Lorentz gauge, which is equivalent to adopting the
coordinate system in which \[ \bar{h}^{\mu \alpha}_{, \alpha} = 0 \]
(i.e. the metric with the divergence of perturbations equal to zero), thus
\begin{equation}
  \label{eq:13}
  - \bar{h}_{\mu \nu, \alpha}^{,\alpha} = 16 \pi T_{\mu \nu}
\end{equation}

\subsection{Solutions in free space}
\label{sec:solutions-free-space}

Solutions in free space will be solutions of 
\begin{equation}
  \label{eq:14}
   \eta^{\alpha \alpha } \bar{h}_{\mu \nu, \alpha \alpha} = \bar{h}_{\mu \nu, \alpha}^{,\alpha} = 0
\end{equation}
which written in slightly more familiar notation is
\begin{equation}
  \label{eq:15}
  \qty( - \pdv[2]{t} + \nabla^2 ) \bar{h}_{\mu \nu} = 0
\end{equation}
moving out of geometrised units, setting $\eta^{00} = -c^{-2}$, then
\begin{equation}
  \label{eq:16}
   \qty( - \pdv[2]{t} + c^2\nabla^2 ) \bar{h}_{\mu \nu} = 0
\end{equation}
This has the form of a wave equation.

\section{Plane wave solutions}
\label{sec:plane-wave-solutions}

\section{Free particles}
\label{sec:free-particles}

\section{Gravitational wave amplitude}
\label{sec:grav-wave-ampl}

\section{Quadrupolarity}
\label{sec:quadrupolarity}



%%% Local Variables: 
%%% mode: latex
%%% TeX-master: "../project"
%%% End: 
