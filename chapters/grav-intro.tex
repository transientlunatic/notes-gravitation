
\section{Thought experiments on gravitation}
\label{sec:thought-exper-grav}

\subsection{Gravitational redshift}
\label{sec:grav-redsh}

Imagine a particle of mass $m$ dropping through a distance $h$; the
particle starts with energy $E=m$ and ends with energy $E=m+mgh$. We
then send this back up to the first position, where it has energy
$E'$. We know $E'=m$, since we'd have a way of making free energy
otherwise! Thus
\[ E' = m = \frac{E}{1+gh} \] Hence, travelling in a gravitational
field causes photons to become redshifted.

\subsection{Schild's photons}
\label{sec:schilds-photons}

Imagine a photon with frequency $f$ fired from a point $A$ to a point
$B$ directly above it in a gravitational field. The photon will be
redshifted to a frequency $f'$, and after some set number of periods
$n$ another photon is sent from a point (in spacetime) $A'$ to
$B'$. The intervals $AB$ and $A'B'$ are the same in all frames, but
$AA'$ and $BB'$ will vary due to the presence of a gravitaional field.

\subsection{The falling lift}
\label{sec:falling-lift}

In special relativity an inertial frame is one where Newton's laws
hold, so particles not affected by an external force move in straight
lines. Consider the situation of a locally inertial frame, where there
are no graviational forces. This occurs in free-fall. Objects in a
locally inertial frame stay at rest; if the inertial frame is
accelerated then the objects won't be---they will increase their speed
towards the floor of the frame at the same rate as the
accleration. This is similar to the observation that all objects fall
at the same rate in a gravitational field; Einstein supposed that this
was not a coincidence, and postulated that:
\begin{quotation}
  Uniform gravitational fields are equivalent to frames that
  accelerate uniformly relative to inertial frames. (Weak equivalence
  principle).
\end{quotation}
Consider a beam of light shining horizontally across an inertial
frame; the beam will end up at a point horizontally opposite where
it's emitted, but as the frame is moving in the field in will have
moved relative to the torch, and in the frame the beam will appear to
be bent by the field.

\section{Relativity and gravitation}
\label{sec:relat-grav}

\subsection{Tides and geodesic deviation}
\label{sec:tides-geod-devi}

Consider two particles, $A$ and $B$, falling towards earth; they start
off level at a height $z(t)$ from the centre of the earth, and
separated by a distance $\xi(t)$. $z$ and $\xi$ are proportional, such
that
\[ \xi(t) = k z(t) \] The force on the particle of mass $m$ at an
altitude $z$ due to gravity is
\begin{equation}
  \label{eq:74}
  F = G \frac{M m }{z^2}
\end{equation}
and so
\begin{equation}
  \label{eq:75}
  \dv[2]{\xi}{t} = k \dv[2]{z}{t} = -k \frac{F}{m} = -k \frac{GM}{z^2} = - \xi \frac{GM}{z^3}
\end{equation}
Thus, as the particles fall they move towards each other.

\subsection{There is no universal inertial frame}
\label{sec:there-no-universal}

In special relativity inertial frames have an infinite extent, but
consider two observers free-falling on opposite sides of the
earth. Each is inertial, but the second derivative of their mutual
separation is non-zero, so special relativity cannot explain what
happens in one of the frames relative to the other. Clearly it isn't
then possible for us to, for example, observe matter falling into a
black hole and understand it with special relativity.


The theory of General Relativity introduces a principle of general
covariance:
\begin{quotation}
  All physical laws must be invariant under all coordinate
  transformations.
\end{quotation}
Thus the approach of general relativity is to express all physics in a
coordinate-independent geometrical description.

\section{Natural units}
\label{sec:natural-units}

In special relativity it is customary to use natural units, where
distance and time are both measured in metres, but in general
relativity we extend this to mass, to measure it in metres also. These
are also known as geometric, or geometrised units.



%%% Local Variables: 
%%% mode: latex
%%% TeX-master: "../project"
%%% End: 
